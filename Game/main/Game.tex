% !TEX program = xelatex
\documentclass[]{article}
\usepackage{float}
\usepackage{graphicx}
\usepackage[svgnames]{xcolor} 
\usepackage{fancyhdr}
\usepackage{fancyvrb}
\usepackage{forest}
\usepackage{tocloft}
\usepackage[hidelinks]{hyperref}
\usepackage{enumitem}
\usepackage[many]{tcolorbox}
\usepackage{listings }
\usepackage[a4paper, total={6in, 8in} , top = 2cm,bottom = 4cm]{geometry}
%\usepackage[a4paper, total={6in, 8in}]{geometry}
\usepackage{afterpage}
\usepackage{amssymb}
\usepackage{pdflscape}
\usepackage{textcomp}
\usepackage{xecolor}
\usepackage{rotating}
\usepackage[Kashida=on,KashidaXBFix=on]{xepersian}
\usepackage[T1]{fontenc}
\usepackage{tikz}
\usepackage[utf8]{inputenc}
\usepackage{PTSerif} 
\usepackage{seqsplit}
\usepackage{changepage}


\usepackage{listings}
\usepackage{xcolor}
\usepackage{sectsty}

\setcounter{secnumdepth}{0}
 
\definecolor{codegreen}{rgb}{0,0.6,0}
\definecolor{codegray}{rgb}{0.5,0.5,0.5}
\definecolor{codepurple}{rgb}{0.58,0,0.82}
\definecolor{backcolour}{rgb}{0.95,0.95,0.92}
\definecolor{blanchedalmond}{rgb}{1.0, 0.92, 0.8}
\definecolor{brilliantlavender}{rgb}{0.96, 0.73, 1.0}
 
\NewDocumentCommand{\codeword}{v}{
\texttt{\textcolor{blue}{#1}}
}
\lstset{language=java,keywordstyle={\bfseries \color{blue}}}

\lstdefinestyle{mystyle}{
    backgroundcolor=\color{backcolour},   
    commentstyle=\color{codegreen},
    keywordstyle=\color{magenta},
    numberstyle=\tiny\color{codegray},
    stringstyle=\color{codepurple},
    basicstyle=\ttfamily\normalsize,
    breakatwhitespace=false,         
    breaklines=true,                 
    captionpos=b,                    
    keepspaces=true,                 
    numbers=left,                    
    numbersep=5pt,                  
    showspaces=false,                
    showstringspaces=false,
    showtabs=false,                  
    tabsize=2
}

\lstset{style=mystyle}

 \settextfont[BoldFont={XB Zar bold.ttf}]{XB Zar.ttf}


\setlatintextfont[Scale=1.0,
 BoldFont={LiberationSerif-Bold.ttf}, 
 ItalicFont={LiberationSerif-Italic.ttf}]{LiberationSerif-Regular.ttf}





\newcommand{\inputsample}[1]{
    ~\\
    \textbf{ورودی نمونه}
    ~\\
    \begin{tcolorbox}[breakable,boxrule=0pt]
        \begin{latin}
            \large{
                #1
            }
        \end{latin}
    \end{tcolorbox}
}

\newcommand{\outputsample}[1]{
    ~\\
    \textbf{خروجی نمونه}

    \begin{tcolorbox}[breakable,boxrule=0pt]
        \begin{latin}
            \large{
                #1
            }
        \end{latin}
    \end{tcolorbox}
}

\newtcolorbox{mybox}[2][]{colback=red!5!white,
colframe=red!75!black,fonttitle=\bfseries,
colbacktitle=red!85!black,enhanced,
attach boxed title to top center={yshift=-2mm},
title=#2,#1}

\newenvironment{changemargin}[2]{%
\begin{list}{}{%
\setlength{\topsep}{0pt}%
\setlength{\leftmargin}{#1}%
\setlength{\rightmargin}{#2}%
\setlength{\listparindent}{\parindent}%
\setlength{\itemindent}{\parindent}%
\setlength{\parsep}{\parskip}%
}%
\item[]}{\end{list}}


\definecolor{foldercolor}{RGB}{124,166,198}
\definecolor{sectionColor}{HTML}{ff5e0e}
\definecolor{subsectionColor}{HTML}{008575}

\definecolor{listColor}{HTML}{00d3b9}

\definecolor{umlrelcolor}{HTML}{3c78d8}

\definecolor{subsubsectionColor}{HTML}{3c78d8}

\defpersianfont\authorFont[Scale=0.9]{XB Zar bold.ttf}

\defpersianfont\titr[Scale=1.5]{Lalezar-Regular.ttf}

\defpersianfont\fehrest[Scale=1.2]{Lalezar-Regular.ttf}

\defpersianfont\fehrestTitle[Scale=3.0]{Lalezar-Regular.ttf}

\defpersianfont\fehrestContent[Scale=1.2]{XB Zar bold.ttf}


\sectionfont{\color{sectionColor}}  % sets colour of sections
\subsectionfont{\color{subsectionColor}}  % sets colour of sections
\subsubsectionfont{\color{subsubsectionColor}}


\renewcommand{\labelitemii}{$\circ$}


\renewcommand{\baselinestretch}{1.1}


\renewcommand{\contentsname}{فهرست}

\renewcommand{\cfttoctitlefont}{\fehrestTitle}


\renewcommand\cftsecfont{\color{sectionColor}\fehrestContent\selectfont}
\renewcommand\cftsubsecfont{\color{subsectionColor}\fehrestContent\selectfont}
\renewcommand\cftsubsubsecfont{\color{subsubsectionColor}\fehrestContent\selectfont}
%\renewcommand{\cftsecpagefont}{\color{sectionColor}}

\setlength{\parskip}{1.2pt}

\begin{document}


%%% title pages
\begin{titlepage}
\begin{center}

\textbf{ \Huge{به نام خدا} }
        
\vspace{0.2cm}

\includegraphics[width=0.4\textwidth]{sharif1.png}\\
\vspace{0.2cm}
\textbf{ \Huge{\emph درس برنامه‌سازی پیشرفته} }\\
\vspace{0.25cm}
\textbf{ \Large{توضیحات بازی} }
\vspace{0.2cm}
       
 
      \large \textbf{دانشکده مهندسی کامپیوتر}\\\vspace{0.1cm}
    \large   دانشگاه صنعتی شریف\\\vspace{0.2cm}
       \large   ﻧﯿﻢ سال دوم 01-00 \\\vspace{0.10cm}
      \noindent\rule[1ex]{\linewidth}{1pt}
استاد:\\
    \textbf{{دکتر محمدامین فضلی}}



    \vspace{0.20cm}


\end{center}
\end{titlepage}
%%% title pages


%%% header of pages and TOC
\tableofcontents
\newpage
\pagestyle{fancy}
\fancyhf{}
\fancyfoot{}
\cfoot{\thepage}
\lhead{فاز اول}
\rhead{\includegraphics[width=0.1\textwidth]{sharif.png}\\
دانشکده مهندسی کامپیوتر
}
\chead{پروژه برنامه‌سازی پیشرفته}
\renewcommand{\headrulewidth}{2pt}

\KashidaOff
\newpage

 \Large \textbf{\\\\}


\section*{{\titr توضیح بخش‌های مختلف بازی}}
\addcontentsline{toc}{section}{{\fehrestContent توضیح بخش‌های مختلف بازی}}

\subsection*{{\titr مقدمه}}
\addcontentsline{toc}{subsection}{{\fehrestContent مقدمه}}
این بازی یک بازی استراتژیک، چند نفره(multiplayer) و نوبتی\lr{(turn based)} است. هر بازی با حداقل 2 بازیکن شروع می‌شود. در ابتدای بازی هر کدام از بازیکنان یک تمدن(civilization) را انتخاب کرده و بازی را شروع می‌کنند. هر بازی روی نقشه‌ای متناسب با تعداد بازیکنان آغاز می‌شود. در این بازی هر کدام از بازیکنان کنترل یک تمدن را بر عهده دارند و به دنبال گسترش تمدن خود و نبرد با دیگر تمدن‌ها هستند. هر نوبت(turn) در این بازی به این صورت است که به یک ترتیب از قبل تعیین شده، تمام بازیکنان نوبتشان را بازی می‎کنند و وقتی تمام بازیکنان یک نوبت بازی کردند همین روند تکرار می‌شود. در ابتدای بازی هر کدام از بازیکنان در نقطه‌ای تصادفی در نقشه بازی شروع می‌کنند و شهر خود را احداث می‌کنند. در این بازی هر کدام از بازیکنان می‌توانند نیروهای مختلفی بسازند و با استفاده از آن‌ها به کشف بقیه نقشه که در ابتدای بازی برایشان پنهان است بپردازند یا به نبرد بقیه بازیکنان بروند. هر کدام از بازیکنان می‌توانند با دستیابی به تکنولوژی‌های مختلف، نیروهای پیشرفته‌تر و قدرتمند‌تری بسازند. هدف در این بازی گسترش تمدن خود(با ساخت شهر‌های جدید یا تصرف شهر‌های دیگر بازیکنان) و شکست دادن بقیه تمدن‌ها است و زمانی بازی پایان می‌یابد که تمام پایتخت‌های تمدن‌های دیگر به تصرف یک تمدن درآمده باشد که آن تمدن برنده بازی خواهد شد. در ادامه بیشتر با قوانین بازی آشنا خواهید شد.
% HMMMMM https://s6.uupload.ir/files/photo_2022-04-07_20-54-38_0ey1.jpg

\subsection*{{\titr Civilizations}}
\addcontentsline{toc}{subsection}{{\fehrestContent Civilizations}}

\subsubsection*{{\titr تمدن‌ها و رهبرها}}
\addcontentsline{toc}{subsubsection}{{\fehrestContent تمدن‌ها و رهبرها}}
هر تمدن در بازی یکتا می باشد. تسلط یافتن بر نقاط قوت یک تمدن یا بهره برداری از نقاط ضعف دشمنان یکی از چالش برانگیزترین بخش های بازی می باشد. وظیفه شما این است که تمدن خود را رهبری کنید و آن را به سمت پیشرفت جلو ببرید.

\subsection*{{\titr War of Fog}} % I can't use /lr here
\addcontentsline{toc}{subsection}{{\fehrestContent War of Fog}}
در بازی همه نقاط تا وقتی که آن هارا کشف نکنید نمایان نیستند. منظور از نمایان نبودن، مشخص نبودن نوع زمین،  این که چه تمدنی صاحب آن زمین است، یا هر نوع اطلاعات دیگر از قبیل نیرو های حاضر در آن و\dots است. شما می‌توانید اطلاعات موجود در هر کاشی را در نقشه‌ی بازی به طور نمادین نشان دهید و یا اطلاعات را با انتخاب آن کاشی به کاربر نمایش دهید.
\subsubsection*{{\titr وضعیت‌های مختلف دانش از هر کاشی}}
\addcontentsline{toc}{subsubsection}{{\fehrestContent وضعیت‌های مختلف دانش از هر کاشی}}
\begin{itemize}
	\item \textbf{شفاف:} 
	اگر یک کاشی در محدوده‌ی دید یکی از شهرها یا واحدهای شما باشد، آن کاشی برای شما شفاف است، به این معنی که می‌توانید تمام ویژگی‌های این کاشی اعم از واحدهای موجود در آن، منابعی که تکنولوژی شما اجازه‌ی مشاهده‌ی آن را می‌دهد و\dots را مشاهده کنید.
	\item \textbf{مشخص:}
	اگر وضعیت کاشی‌ای زمانی برای شما شفاف شده باشد، اما دیگر در این وضعیت نباشد، شما همچنان ویژگی‌های زمین آن کاشی را مشاهده می‌کنید. حتی اگر در گذشته در این کاشی مربوط به شهری بوده است، می‌توانید از این وضعیت آگاه شوید، اما تغییرات جدید این کاشی، پیشرفت‌های موجود در این کاشی، شهرهای تازه‌ساخته‌شده واحدهای موجود در این کاشی را نمی‌بینید.
	\item \textbf{\lr{Fog of war}:} شما از کاشی‌هایی که تاکنون برایتان شفاف نشده‌اند هیچ اطلاعاتی به جز موقعیت مکانی آن‌ها ندارید.
\end{itemize}
\subsubsection*{{\titr چه چیز‌هایی قابل مشاهده هستند؟}}
\addcontentsline{toc}{subsubsection}{{\fehrestContent چه چیز‌هایی قابل مشاهده هستند؟}}
کاشی‌های متعلق به سرزمین شما (محدوده مشخص شده توسط هر شهر) همیشه قابل مشاهده هستند. همچنین کاشی‌هایی که فاصله 1 از مرز های قلمرو شما دارند نیز قابل مشاهده هستند. اکثر نیرو‌ها میتوانند تا 2 کاشی فاصله از خود را مشاهده کنند. (به جز زمین های خاصی که در ادامه به آن‌ها اشاره می‌کنیم.)

همچنین دقت کنید اگر نیروی شما حرکت کند و دیگر آن کاشی در محدوده‌ی دید نیرو نباشد وضعیت کاشی به مشخص تغییر پیدا می‌کند.
\subsubsection*{{\titr چه چیزهایی دید نیرو‌ها را کاهش می‌دهند؟}}
\addcontentsline{toc}{subsubsection}{{\fehrestContent چه چیزهایی دید نیرو‌ها را کاهش می‌دهند؟}}
کوه‌ها جلوی دید نیرو‌ها را می‌گیرند به این معنی که نیروها آن سوی کوه را نمی‌توانند مشاهده کنند.
جنگل ها، کوه‌ها و تپه ها زمین های block کننده هستند. منظور از block کننده این است که خود آن‌ها را میتوانید مشاهده کنید اما آن سوی آن‌ها قابل مشاهده نیست. اگر نیروی شما روی این کاشی‌ها باشند، دیگر این محدودیت را ندارد و می‌تواند آن سوی زمین های block کننده را ببیند.

\newpage
\subsection*{{\titr Info}} % (Hirbod): Translation of info
\addcontentsline{toc}{subsection}{{\fehrestContent Info}}

\subsubsection*{{\titr اطلاعات}} % (Hirbod): Translation of info
\addcontentsline{toc}{subsubsection}{{\fehrestContent اطلاعات}}
صفحه بازی شامل اطلاعاتی می‌باشد. با استفاده از این اطلاعات میتوانیم بفهمیم تا اینجای بازی چقدر موفق بوده‌ایم. با استفاده از دکمه هایی در نقشه اصلی یا کلیدهای میانبر می توانیم به آنها دسترسی داشته باشیم.
\begin{itemize}
	\item \textbf{اطلاعات کاوش:} این بخش، پروژه کاوش فعلی شما را نشان میدهد . همچنین نشان میدهد چند نوبت برای کاوش باقی مانده است و در نهایت چه چیزی برای شما باز خواهد شد. برای اطلاعات بیشتر بخش تکنولوژی را مطالعه فرمایید. 
	\item \textbf{پنل لیست یگان‌ها:} در این صفحه می توانید تمام یگان های خود و وضعیت آنها را ببینید. در این بخش باید بتوانید یک یگان را فعال کنید. برای اطلاعات بیشتر میتوانید به بخش یگان ها مراجعه کنید.
	از این بخش میتوانید به بخش بررسی کلی نظامی بروید.
	\item \textbf{پنل لیست شهرها:} در این بخش میتوانید لیست شهرها را ببینید و باید این امکان را داشته باشید تا بتوانید به صفحه هر شهر بروید. برای اطلاعات بیشتر بخش شهرها را مطالعه کنید.
	همچنین باید توانایی ورود به صفحه بررسی کلی اقتصادی را داشته باشید.
	\item \textbf{پنل اطلاعات دیپلماسی:} در این صفحه می توانید امتیاز بازی را مشاهده کنید. همچنین در این بخش میتوانید برای انجام دیپلماسی با تمدن‌های شناخته شده، اقدام کنید.
	\item \textbf{صفحه پیشرفت پیروزی:} این صفحه نشان دهنده پیشرفت فعلی شما در مسیرهای مختلف پیروزی در بازی است.
	\item \textbf{صفحه جمعیت شناسی:} در این بخش میتوانید به اطلاعات زیادی درباره تمدن خود دست یابید. از قبیل : اندازه ، ثروت ، ارتش ، خروجی و \dots . با این اطلاعات شما میتوانید تمدن خود را با سایرین مقایسه کنید و رتبه خود را به همراه امتیاز در مقایسه با بهترین ، بدترین و میانگین امتیازات مشاهده کنید.
	\item \textbf{تاریخچه اطلاعیه‌ها:} در این بخش میتوانید تمامی اطلاعیه هایی که در طول بازی برای شما ارسال شده است را ببینید. همچنین امکان این را دارید که ببینید این اطلاعیه مربوط به دور چند است. این بخش برای اطمینان از خواندن تمامی اطلاعیه ها و از دست ندادن آن ها میباشد.
	\item \textbf{بررسی کلی نظامی:} این صفحه تمام واحدها و یگان های شما را نمایش میدهد.
	\item \textbf{بررسی کلی اقتصادی:} در این بخش میتوانید اطلاعات جامع تری از شهر های خود را مشاهده کنید. اطلاعاتی از قبیل : جمعیت، قدرت دفاعی، خروجی غذا، علم، طلا، بهره‌وری و آنچه هم اکنون در حال ساخت است و مدت زمانی که تا پایان ساخت آن باقی مانده است.
	می‌توان از این قسمت به صفحه‌ی هر شهر رفت.
	\item \textbf{بررسی کلی دیپلماسی:} در این بخش میتوانید وضعیت دیپلماتیک فعلی خود با دیگر تمدن‌ها را مشاهده کنید. 
	\item \textbf{صفحه تاریخچه معامله:} در این بخش میتوانید معاملات دیپلماتیک در حال انجام و همچنین اطلاعات مهمی درباره معاملات قبلی را مشاهده کنید.
\end{itemize}


\subsection*{{\titr Features and Terrain}}
\addcontentsline{toc}{subsection}{{\fehrestContent Features and Terrain}}

\subsubsection*{{\titr زمین}}
\addcontentsline{toc}{subsubsection}{{\fehrestContent زمین}}
در این بازی، جهان از «کاشی‌»های شش ضلعی تشکیل شده‌است. (همچنین گاهی اوقات به عنوان هگز نیز شناخته می‌شوند.) این کاشی‌ها در انواع مختلفی از «نوع زمین» وجود دارند - کویر، دشت، علفزار، تپه و غیره - و بسیاری نیز شامل «ویژگی‌هایی» مانند بیشه‌ها و جنگل‌ها هستند. این عنصرها کمک می‌کنند تا میزان مفید بودن کاشی را برای شهر نزدیکش تشخیص دهیم؛ یا آنکه چقدر عبور از کاشی سخت یا آسان است. زمین و ویژگی‌های هر کاشی ممکن است اثرات مهمی بر مبارزه‌هایی بگذارد که در آن کاشی رخ می‌دهند.

\subsubsection*{{\titr منابع}}
\addcontentsline{toc}{subsubsection}{{\fehrestContent منابع}}
منابع منشاء غذا، بهره‌وری و طلا هستند؛‌ برخی از آن‌ها هم امتیازات ویژه‌ای را برای تمدن فراهم می‌کنند. آن‌ها در هگزهای مشخصی پدید می‌آیند. بعضی از آن‌ها در ابتدای بازی قابل مشاهده هستند وبرخی دیگر قبل آنکه قابل مشاهده باشند به مالکیت یک سری تکنولوژی‌های مشخص نیاز دارند. می‌توانید برای بررسی بیشتر، بخش منابع را بررسی کنید.
\\

\subsubsection*{{\titr توضیح مقدارهای هر زمین}}
\addcontentsline{toc}{subsubsection}{{\fehrestContent توضیح مقدارهای هر زمین}}

\begin{itemize}
	\item \textbf{بازده شهر:} به میزان غذا، طلا یا بهره‌وری‌ای گفته می‌شود که یک شهر می‌تواند از یک کاشی توسعه نیافته از آن نوع دریافت کند.
	\item \textbf{هزینه‌ی حرکت:} هزینه (با واحد MP یا امتیاز حرکت) برای وارد شدن به یک نوع کاشی.
	\item \textbf{تغییرات مبارزه:} تغییر در قدرت حمله یا دفاع از یک یگان که در آن کاشی قرار می‌گیرد.
\end{itemize}
\subsubsection*{{\titr انواع زمین}}
\addcontentsline{toc}{subsubsection}{{\fehrestContent انواع زمین}}
\begin{itemize}
	\item \textbf{کویر:} کلاً هگزهای کویر به طور قابل توجهی بی‌فایده هستند. آن‌ها هیچ فایده‌ای برای شهرها فراهم نمی‌کنند (البته مگر اینکه کویر شامل واحد یا منابعی باشد)، و یگان‌هایی که آن‌ها را اشغال می‌کنند یک مجازات دفاعی  در مبارزه دریافت می‌کنند.\\
	غذا: ۰\\
	تولید: ۰\\
	طلا: ۰\\
	تغییرات مبارزه: ۳۳-٪\\
	هزینه‌ی حرکت: ۱
	\item \textbf{علفزار:} در کل علفزارها بیش‌ترین غذا را در بین انواع زمین تولید می‌کنند. شهرهای ساخته‌شده در نزدیکی علفزارها سریع‌تر از شهرهای ساخته‌شده در مکان‌های دیگر رشد خواهند کرد. مشکل اصلی این کاشی‌ها جریمه‌ی دفاعی هست که یک یگان ناآماده ممکن است بگیرد اگر حمله کند.\\
	غذا: ۲\\
	تولید: ۰\\
	طلا: ۰\\
	تغییرات مبارزه: ۳۳-٪\\
	هزینه‌ی حرکت: ۱
	\item \textbf{تپه‌ها:} کشاورزی و حرکت در آن‌ها سخت است، ولی آن‌ها امتیازات دفاعی خوبی فراهم می‌کنند و منابع مختلف بسیاری در آن یافت می‌شود. علاوه بر این، یگان‌های بالای تپه می‌توانند فراز «بلاک‌های مسدودکننده» را ببینند.\\
	غذا: ۰\\
	تولید: ۲\\
	طلا: ۰\\
	تغییرات مبارزه: ۲۵+٪\\
	هزینه‌ی حرکت: ۲
	\item \textbf{کوه:} کوه‌ها برآمدگی‌های بلند زمین هستند. عبور از آن‌ها ناممکن است. آن‌ها برای یک تمدن مفید نیستند، مگر به عنوان موانعی برای تهاجم تمدن‌های دیگر.\\
	غذا: ۰\\
	تولید: ۰\\
	طلا: ۰\\
	تغییرات مبارزه: ۰٪\\
	هزینه‌ی حرکت: غیرقابل‌عبور
	\item \textbf{اقیانوس:} هگزهای اقیانوس، هگزهای آب‌های عمیق هستند. که امکان عبور از روی آن‌ها وجود ندارد.\\
	غذا: ۰\\
	تولید: ۰\\
	طلا: ۰\\
	تغییرات مبارزه: هیچ\\
	هزینه‌ی حرکت: غیرقابل‌عبور
	\item \textbf{دشت‌ها:} دشت‌ها ترکیبی از غذا و تولید را برای یک شهر نزدیک فراهم می‌کنند. شهری که توسط دشت‌ها احاطه شده‌است، آهسته‌تر از شهری که در علفزار است رشد می‌کند ولی بسیار تولید بیش‌تری خواهد داشت.\\
	غذا: ۱\\
	تولید: ۱\\
	طلا: ۰\\
	تغییرات مبارزه: ۳۳-٪\\
	هزینه‌ی حرکت: ۱
	\item \textbf{برف:} برف کاملاً بدون تولید است و هیچ مزیت غذایی یا تولیدی برای شهر نزدیک ندارد. البته ممکن است هگز برف شامل منابع مفیدی باشد، اما در غیر این صورت آنها فقط سرد و بی‌ثمر هستند.\\
	غذا: ۰\\
	تولید: ۰\\
	طلا: ۰\\
	تغییرات مبارزه: ۳۳-٪\\
	هزینه‌ی حرکت: ۱
	\item \textbf{توندرا:} توندرا سرزمین نیمه یخ‌زده‌ای است که در آب و هوای سردتر جهان یافت می‌شود. توندرا کم فایده‌تر از دشت یا علفزار است ولی کمی بهتر از کویر. هیچ‌کس شهری در توندرا نمی‌سازد مگر آنکه از منابع ناامید باشند یا آنکه هیچ جای دیگری برای رفتن پیدا نکنند.\\
	غذا: ۱\\
	تولید: ۰\\
	طلا: ۰\\
	تغییرات مبارزه: ۳۳-٪\\
	هزینه‌ی حرکت: ۱
\end{itemize}
\subsubsection*{{\titr ویژگی‌ها}}
\addcontentsline{toc}{subsubsection}{{\fehrestContent ویژگی‌ها}}
ویژگی‌ها، عناصر زمین یا پوشش گیاهی هستند که در یک هگز پدید می‌آیند، بالای زمین هگز. (برای مثال هگز علفزار ممکن است که بیشه یا مرداب در درون خود داشته باشد.) ویژگی‌ها، بهره‌وری یک هگز را تغییر می‌دهند و همچنین ممکن است مقدار «امتیاز حرکتی» (MP) که یک یگان هنگام ورود به هگز هزینه می‌کند را تغییر دهد. ویژگی‌ها همچنین ممکن است که امتیازات یا جریمه‌های مبارزه‌ی دفاعی را برای یگانی که هگز را اشغال می‌کند فراهم کند.
\\\noindent \textbf{مقدارهای ویژگی:} مانند زمین، ویژگی‌ها هم مقدارهایی دارند که بازده شهرها، حرکت و مبارزه‌ی آن‌ها را مشخص می‌کنند.
\newpage
\begin{itemize}
	\item \textbf{جلگه:} جلگه‌ها مناطق مجاور رودخانه‌ها هستند. این نوع خانه‌ها تنها در مجاورت رودخانه یافت می‌شوند.\\
	غذا: ۲\\
تولید: ۰\\
طلا: ۰\\
تغییرات مبارزه: ۳۳-٪\\
هزینه‌ی حرکت: ۱\\
	\item \textbf{جنگل:} کاشی‌های جنگل مستقل از نوع زمین زیرشان دارای غذا و تولید ۱ هستند.\\
	غذا: ۱\\
	تولید: ۱\\
	طلا: ۰\\
	تغییرات مبارزه: +۲۵٪\\
	هزینه‌ی حرکت: ۲
	\item \textbf{یخ:} یخ‌ها کاملا بدون کاربرد و غیرقابل‌عبور هستند.\\
	غذا: ۰\\
	تولید: ۰\\
	طلا: ۰\\
	تغییرات مبارزه: ۰٪\\
	هزینه‌ی حرکت: غیرقابل‌عبور
	\item \textbf{جنگل انبوه:} این خانه‌ها برای تولید نامناسب هستند، با بریدن جنگل انبوه، این خانه‌ها به دشت تبدیل می‌شوند.\\
	غذا: ۱\\
	تولید: ۱-\\
	طلا: ۰\\
	تغییرات مبارزه: +۲۵٪\\
	هزینه‌ی حرکت: ۲
	\item \textbf{مرداب:} مرداب‌ها در تولید غذا تاثیر منفی دارند.\\
	غذا: ۱-\\
	تولید: ۰\\
	طلا: ۰\\
	تغییرات مبارزه: ۳۳-٪\\
	هزینه‌ی حرکت: ۲
	\item \textbf{واحه:} واحه‌ها آبادی‌هایی میان کویر هستند (پس خارج از کویر واحه نداریم.)\\
	غذا: ۳\\
	تولید: ۰\\
	طلا: ۱\\
	تغییرات مبارزه: ۳۳-٪\\
	هزینه‌ی حرکت: ۱
\end{itemize}
%dutchman
\subsection*{{\titr Resources}}
\addcontentsline{toc}{subsection}{{\fehrestContent Resources}}

\subsubsection*{{\titr منابع}}
\addcontentsline{toc}{subsubsection}{{\fehrestContent چه چیز‌هایی قابل مشاهده هستند؟}}
منابع، منشا غذا، حاصلخیزی و امتیازات دیگر برای یک تمدن هستند. ثروت و قدرت شما تا حد زیادی به تعداد و نوع منابعی که کنترل می‌کنید بستگی دارد. برای بهره‌برداری از یک منبع، آن منبع باید در مرزهای تمدن شما قرار داشته‌باشد و شما نیز باید پیشرفت لازم را روی آن کاشی از نقشه اعمال کنید. (برای مثال با اعمال پیشرفت «کشت‌وکار» می‌توانید از منبع «موز»ی که در قلمرو شما قرار دارد استفاده کنید.)
سه نوع منبع در بازی وجود دارد: امتیازی، لوکس و استراتژیک که هر سه برای شهرهای اطراف سودمند هستند. البته منابع لوکس و استراتژیک فواید مهم دیگری نیز دارند.
از آن جایی که ممکن است در قلمرو خود به انواعی از منابع دست پیدا نکنید، می‌توانید بعضی منابع را با دیگر تمدن‌ها مبادله کنید.

\subsubsection*{{\titr منابع امتیازی}}
\addcontentsline{toc}{subsubsection}{{\fehrestContent منابع امتیازی}}
این منابع خروجی غذا و طلا را در یک کاشی افزایش می‌دهند. امتیازها را نمی‌توان با تمدن‌های دیگر مبادله کرد.\\
\begin{itemize}
\item موز\\
غذا: +1\\
تولید‌: صفر\\
طلا: صفر\\
در این مکان‌ها پیدا می‌شود: جنگل انبوه\\
پیشرفت لازم برای بهره‌برداری: کشت‌وکار
\item گاو\\
غذا: +1\\
تولید‌: صفر\\
طلا: صفر\\
در این مکان‌ها پیدا می‌شود: علفزار\\
پیشرفت لازم برای بهره‌برداری: چراگاه
\item آهو\\
غذا: +1\\
تولید‌: صفر\\
طلا: صفر\\
در این مکان‌ها پیدا می‌شود: جنگل، توندرا، تپه\\
پیشرفت لازم برای بهره‌برداری: کمپ
\item گوسفند\\
غذا: +1\\
تولید‌: صفر\\
طلا: صفر\\
در این مکان‌ها پیدا می‌شود: دشت‌، علفزار، کویر، تپه\\
پیشرفت لازم برای بهره‌برداری: چراگاه
\item گندم\\
غذا: +1\\
تولید‌: صفر\\
طلا: صفر\\
در این مکان‌ها پیدا می‌شود: دشت، جلگه\\
پیشرفت لازم برای بهره‌برداری: مزرعه
\end{itemize}


\subsubsection*{{\titr منابع لوکس}}
\addcontentsline{toc}{subsubsection}{{\fehrestContent منابع لوکس}}
این نوع از منابع شادی را در تمدن شما بیشتر می‌کنند. همچنین خروجی‌های یک کاشی را مقدار کمی افزایش می‌دهند. دسترسی به یک نوع خاص از منابع لوکس فقط برای بار اول باعث افزایش شادی می‌شود. البته پیدا کردن چند منبع لوکس از یک نوع همچنان ارزشمند است زیرا می‌توانید آن‌ها را با دیگر تمدن‌ها مبادله کنید. برای مثال دو منبع ابریشم به اندازه یک منبع ابریشم شادی را افزایش می‌دهند ولی اگر یکی از آن‌ها را با یک منبع شکر از تمدنی دیگر مبادله کنید، منبع شکر می‌تواند به شادی قلمرو شما اضافه کند. هر نوع منبع لاکچری ۴ واحد به شادی تمدن شما اضافه می‌کند.
\newpage
\begin{itemize}
\item پنبه\\
غذا: صفر\\
تولید‌: صفر\\
طلا: +2\\
در این مکان‌ها پیدا می‌شود: دشت، کویر، علفزار\\
پیشرفت لازم برای بهره‌برداری: کشت‌وکار
\item رنگ\\
غذا: صفر\\
تولید‌: صفر\\
طلا: +2\\
در این مکان‌ها پیدا می‌شود: جنگل انبوه، جنگل\\
پیشرفت لازم برای بهره‌برداری: کشت‌وکار
\item خز\\
غذا: صفر\\
تولید‌: صفر\\
طلا: +2\\
در این مکان‌ها پیدا می‌شود: جنگل، توندرا\\
پیشرفت لازم برای بهره‌برداری: کمپ
\item سنگ‌های قیمتی\\
غذا: صفر\\
تولید‌: صفر\\
طلا: +3\\
در این مکان‌ها پیدا می‌شود: جنگل انبوه،  توندرا، دشت، کویر،\\ علفزار، تپه\\
پیشرفت لازم برای بهره‌برداری: معدن
\item طلا\\
غذا: صفر\\
تولید‌: صفر\\
طلا: +2\\
در این مکان‌ها پیدا می‌شود: دشت، کویر، علفزار، تپه\\
پیشرفت لازم برای بهره‌برداری: معدن
\item بخور\\
غذا: صفر\\
تولید‌: صفر\\
طلا: +2\\
در این مکان‌ها پیدا می‌شود: دشت، کویر\\
پیشرفت لازم برای بهره‌برداری: کشت‌وکار
\item عاج\\
غذا: صفر\\
تولید‌: صفر\\
طلا: +2\\
در این مکان‌ها پیدا می‌شود: دشت\\
پیشرفت لازم برای بهره‌برداری: کمپ
\item مرمر\\
غذا: صفر\\
تولید‌: صفر\\
طلا: +2\\
در این مکان‌ها پیدا می‌شود: توندرا، دشت، کویر، علفزار، تپه\\
پیشرفت لازم برای بهره‌برداری: معدن سنگ
\item ابریشم\\
غذا: صفر\\
تولید‌: صفر\\
طلا: +2\\
در این مکان‌ها پیدا می‌شود: جنگل\\
پیشرفت لازم برای بهره‌برداری: کشت‌وکار
\item نقره\\
غذا: صفر\\
تولید‌: صفر\\
طلا: +2\\
در این مکان‌ها پیدا می‌شود: توندرا، کویر، تپه\\
پیشرفت لازم برای بهره‌برداری: معدن
\item شکر\\
غذا: صفر\\
تولید‌: صفر\\
طلا: +2\\
در این مکان‌ها پیدا می‌شود: جلگه، مرداب\\
پیشرفت لازم برای بهره‌برداری: کشت‌وکار
\end{itemize}


\subsubsection*{{\titr منابع استراتژیک}}
\addcontentsline{toc}{subsubsection}{{\fehrestContent منابع استراتژیک}}
این منابع در ابتدای بازی قابل رویت نیستند. برای ظاهر شدن آن‌ها روی نقشه ابتدا باید به نوعی خاص از تکنولوژی دسترسی داشته باشید. برای مثال منابع «اسب» پس از داشتن تکنولوژی دامداری و منابع «آهن» پس از داشتن فناوری کار با آهن روی نقشه ظاهر می‌شوند. 
\\
منابع استراتژیک به شما امکان ساخت انواع جدیدی از ساختمان‌‌ها و واحدها را می‌دهند. اگر پیشرفت لازم را روی یک کاشی حاوی منبعی استراتژیک اعمال کنید، تعداد مشخصی از آن منبع به شما داده می‌شود و شما می‌توانید از آن‌ها برای ساخت واحدها و ساختمان‌هایی که به این منبع احتیاج دارند استفاده کنید. برای مثال شما برای ساخت یک واحد شمشیرزن نیاز به یک آهن دارید. بنابراین بدون آهن نمی‌توانید این واحد را بسازید. پس از از بین رفتن یک واحد و ساختمان منابعی که برای ساخت آن‌ها استفاده کرده‌اید به دارایی شما برخواهد گشت.

شما می‌توانید منابع استراتژیک را با دیگر تمدن‌ها مبادله کنید.

\begin{itemize}
\item زغال‌سنگ\\
غذا: صفر\\
تولید‌: +1\\
طلا: صفر\\
در این مکان‌ها پیدا می‌شود: دشت، تپه، علفزار\\
تکنولوژی‌‌ مورد نیاز برای آشکار شدن: نظریه‌ی علمی\\
پیشرفت لازم برای بهره‌برداری: معدن\\
یگان‌های وابسته به این منبع: 	زره‌پوش\\
ساختمان‌های وابسته به این منبع: کارخانه\\

\item اسب\\
غذا: صفر\\
تولید‌: +1\\
طلا: صفر\\
در این مکان‌ها پیدا می‌شود: توندرا، دشت، علفزار\\
تکنولوژی‌‌ مورد نیاز برای آشکار شدن: دامپروری\\
پیشرفت لازم برای بهره‌برداری: چراگاه\\
یگان‌های وابسته به این منبع: 	ارابه‌ی کماندار، سواره نظام\\ همراه، سواره‌ نظام، سوارکاران، قزاق، نیزه‌دار، سپاهی، کماندار\\ شترسوار، شوالیه، سواره نظام ماندکالو\\ 
ساختمان‌های وابسته به این منبع: طویله، سیرک\\

\item آهن\\
غذا: صفر\\
تولید‌: +1\\
طلا: صفر\\
در این مکان‌ها پیدا می‌شود: توندرا، دشت، کویر،  تپه، علفزار، برف\\
تکنولوژی‌‌ مورد نیاز برای آشکار شدن: کار با آهن\\
پیشرفت لازم برای بهره‌برداری: معدن\\یگان‌های
بالیستا، سنگ‌انداز، جنگجوی موهاک شمشیرزن، شمشیرزن با شمشیر بلند، سامورایی، تربوشه\\
ساختمان‌های وابسته به این منبع: کوره‌ی آهنگری\\
\end{itemize}

\newpage
\subsection*{{\titr Actions Units}}
\addcontentsline{toc}{subsection}{{\fehrestContent Actions Units}}

\subsubsection*{{\titr یگان‌ها}}
\addcontentsline{toc}{subsubsection}{{\fehrestContent یگان‌ها}}
به هر چیزی که بتواند حرکت کند، یگان (یا واحد) گفته می‌شود. یگان‌ها ممکن است نظامی یا غیرنظامی (کارگرها و مهاجرین) باشند.

\subsubsection*{{\titr ایجاد یگان‌ها}}
\addcontentsline{toc}{subsubsection}{{\fehrestContent ایجاد یگان‌ها}}
یگان‌ها در شهرها ایجاد می‌شوند. هر یگان، هزینه تولید خاص خود را دارد که برای ایجاد شدن باید توسط آن شهر پرداخت شود. همچنین ممکن است یک یگان نیاز به تکنولوژی خاصی (برای مثال نمی‌توانید قبل از دستیابی به تکنولوژی تیراندازی یگان‌های کماندار بسازید) و یا منابع خاصی (برای مثال برای ساخت شوالیه، باید حتما دسترسی به اسب داشته باشید) داشته باشد.

\subsubsection*{{\titr مشخصات یگان‌ها}}
\addcontentsline{toc}{subsubsection}{{\fehrestContent مشخصات یگان‌ها}}
هر یگان سه مشخصه دارد: سرعت حرکت، قدرت نظامی و ارتقا
\begin{itemize}
\item \noindent \textbf{سرعت حرکت:}
تعداد حرکت‌های یگان‌ها مشخص می‌کند که در هر نوبت چه تعداد کاشی می‌توانند حرکت کنند. برای توضیحات کامل درباره حرکت یگان‌ها به بخش «حرکت» مراجعه کنید.
\item \noindent \textbf{قدرت نظامی:}
قدرت نظامی هر یگان، مشخص‌کننده توانایی او در جنگ‌هاست. قدرت نظامی تمامی یگان‌های غیرنظامی مانند کارگران و مهاجرین صفر است، بدین معنی که با حمله کردن هر یگان نظامی به این یگان‌ها، به اسارت یگان مهاجم درآمده و یا از بین می‌روند.

\end{itemize}


\subsubsection*{{\titr توانایی انحصاری یگان‌ها}}
\addcontentsline{toc}{subsubsection}{{\fehrestContent توانایی انحصاری یگان‌ها}}
برخی یگان‌ها می‌توانند برخی کارها را بهتر از یگان‌های دیگر انجام دهند و یا حتی کارهایی انجام دهند که یگان‌های دیگر قادر به آن نیستند. برای مثال یگان مهاجرین (settler) می‌تواند یک شهر جدید در محل حضور خود توسعه دهد یا یگان‌های تیرانداز می‌توانند از فاصله دور به دشمنان حمله کنند.

\subsubsection*{{\titr حرکت یگان‌ها}}
\addcontentsline{toc}{subsubsection}{{\fehrestContent حرکت یگان‌ها}}
یگان‌ها در کاشی‌های شش‌ضلعی‌شکل حرکت می‌کنند و این حرکت متناسب با کاشی‌ای که به آن می‌روند، تعدادی حرکت از یگان در آن نوبت کم می‌کند. همچنین دو یگان نظامی و یا دو یگان غیرنظامی نمی‌توانند در یک کاشی قرار گیرند، اما یک یگان نظامی و یک یگان غیرنظامی ممکن است در یک کاشی مشترک قرار گیرند. همچنین محدودیت‌هایی در حرکت یگان‌ها وجود دارد. برای مثال یگان‌ها نمی‌توانند به کوه بروند و یا از آن رد شوند. ساختن جاده‌ها موجب افزایش حرکت یگان‌ها می‌شود. برای اطلاعات بیشتر به بخش «حرکت» مراجعه کنید.

\subsubsection*{{\titr نبرد یگان‌ها}}
\addcontentsline{toc}{subsubsection}{{\fehrestContent نبرد یگان‌ها}}
یگان‌های نظامی می‌توانند به همدیگر و همین‌طور به شهرها حمله کنند. در صورتی که یک یگان دقیقا یک فاصله با یگان دیگری داشته باشد، می‌تواند به آن حمله کند. توجه کنید که یگان‌های تیرانداز می‌توانند از فاصله‌ای بیشتر از یک کاشی نیز حمله کنند.
برای اطلاعات بیشتر به بخش «نبرد» مراجعه کنید.

\subsubsection*{{\titr یگان‌های غیرنظامی}}
\addcontentsline{toc}{subsubsection}{{\fehrestContent یگان‌های غیرنظامی}}
یگان‌های غیرنظامی شامل کارگران و مهاجرین می‌شود که هرکدام نقش بزرگی برای تمدن خود ایفا می‌کنند. همانطور که از اسم این یگان‌ها مشخص است، قدرت نظامی آن‌ها صفر است و نمی‌توانند مبارزه کنند و اگر توسط یک یگان نظامی مورد حمله قرار بگیرند، ممکن است اسیر یا نابود شوند. برای مراقبت بیشتر از این یگان‌ها می‌توان یک یگان نظامی در کاشی آن‌ها قرار داد.

\subsubsection*{{\titr یگان‌های مبارز}}
\addcontentsline{toc}{subsubsection}{{\fehrestContent یگان‌های مبارز}}
یگان‌های مبارز یا نظامی به چند دسته تقسیم می‌شوند.
\begin{itemize}

\item \noindent \textbf{یگان‌های با سلاح سرد:}
این یگان‌ها (مانند جنگجو، نیزه‌دار و …) محدوده حمله کوچکی دارند و تنها می‌توانند به یگان‌هایی حمله کنند که دقیقا یک فاصله از آن‌ها فاصله داشته باشند.
\item \noindent \textbf{یگان‌های تیرانداز:}
این یگان‌ها می‌توانند به یگان‌هایی با بیشتر از یک کاشی فاصله نیز حمله کنند. این که این یگان‌ها تا چه حد دورتر را بتوانند هدف قرار دهند، به ویژگی «محدوده حمله» آن‌ها بستگی دارد و قدرت حمله از راه دور آن‌ها متفاوت است. برای مثال یگان کمان‌دار قدرت حمله از راه دور 7 و قدرت نظامی 4 و محدوده حمله 2 دارد. یعنی این یگان می‌تواند تا 2 کاشی دورتر از کاشی خود را با قدرت 7 مورد هدف قرار دهد. اما اگر به آن حمله شود، با قدرت نظامی خود (4) دفاع می‌کند. توجه کنید که این نوع یگان‌ها همواره با قدرت حمله از راه دور خود به دشمن حمله می‌کنند حتی اگر در فاصله یک کاشی از آن‌ها باشند.
\end{itemize}

\subsubsection*{{\titr لیست عملیات یگان‌ها}}
\addcontentsline{toc}{subsubsection}{{\fehrestContent لیست عملیات یگان‌ها}}
\begin{itemize}
\item \noindent \textbf{حرکت کردن:} 
به یک کاشی دیگر حرکت می‌کند.
\item \noindent \textbf{خوابیدن:} 
با فعال کردن این گزینه، یگان به حالت غیرفعال در می‌آید. یعنی در نوبت‌های بعدی درخواست عملیات جدید نمی‌کند تا زمانی که یک دستور جدید دریافت کند.
\item \noindent \textbf{حالت آماده باش:} 
مشابه حالت خوابیدن عمل خواهد کرد، اما در صورتی که دشمنی در نزدیکی خود رؤیت کند آماده دریافت دستورات می‌شود.
\item \noindent \textbf{تقویت:} 
مشابه حالت خوابیدن اما با این تفاوت که در این حالت، قدرت دفاعی یگان افزایش خواهد یافت.
\item \noindent \textbf{تقویت تا ترمیم کامل:}
مشابه تقویت می‌باشد اما به‌جای اینکه یگان قدرت دفاعی دریافت کند، جان خود را افزایش می‌دهد.
\item \noindent \textbf{مستقر شدن:} 
این گزینه تنها زمانی قابل انجام است که یگان دقیقا درون شهر خودی (در کاشی‌ای که شهر در آن بنا شده) قرار داشته باشد. در این صورت یگان وارد شهر می‌شود و با قدرت نظامی بیشتری از شهر خود دفاع می‌کند.
\item \noindent \textbf{آماده‌سازی برای حمله از راه دور:} 
در ابزار محاصره units Siege قبل از حمله، لازم است که کاشی‌ای را با این گزینه به عنوان هدف مشخص کنند.
\item \noindent \textbf{حمله از راه دور:} 
برای یگان‌هایی که توانایی این کار را داشته باشند استفاده می‌شود تا به کاشی انتخاب‌شده حمله کنند.
\item \noindent \textbf{غارت:} 
پیشرفت (مانند مزرعه، معدن و …) موجود در کاشی فعلی را از بین می‌برد. پس از این دستور تا زمانی که آن کاشی مجدداً تعمیر نشود، قابل استفاده نیست (منابع حاصل از آن کاشی در دسترس نخواهند بود).
\item \noindent \textbf{پیدایش شهر:}
این گزینه تنها برای مهاجرین قابل اجراست و پس از اجرای آن، یک شهر جدید در آن کاشی برای تمدن به وجود می‌آید و مهاجرین نابود می‌شوند.
\item \noindent \textbf{لغو:} 
در صورتی که به اشتباه دستوراتی وارد صف دستورات یگان کرده‌اید، می‌توانید با استفاده از این گزینه دستورات موجود در صف را حذف کنید.
\item \noindent \textbf{بیدار شدن:}
در صورت خواب بودن یگان، آن را بیدار می‌کند.
\item \noindent \textbf{حذف:} 
یگان را از بین می‌برد و در ازای آن ده درصد هزینه‌ی تولید یگان سکه دریافت می‌کند.
\end{itemize}

\subsection*{{\titr Movement}}
\addcontentsline{toc}{subsection}{{\fehrestContent Movement}}
در این بازی، اکثرا در حال حرکت واحدهایتان در دنیا هستید. به عنوان‌ مثال، نیروهای نظامی خود را برای پیداکردن منابع یا جنگ با همسایگان، کارگران‌تان را برای بهبود پستی‌بلندی‌های زمین و نیز ساخت جاده‌ها و \lr{Settler}هایتان را برای شهرسازی به مناطق مناسب حرکت می‌دهید.

\subsubsection*{{\titr قوانین حرکت}}
\addcontentsline{toc}{subsubsection}{{\fehrestContent قوانین حرکت}}

درصورتیکه حرکت به نقطه مقصد مجاز باشد، واحد انتخابی به آنجا حرکت می‌کند. در غیر اینصورت دستور حركت را رد کرده و منتظر دستورات جدید می‌ماند.

\subsubsection*{{\titr حرکت چند نوبتی}}
\addcontentsline{toc}{subsubsection}{{\fehrestContent حرکت چند نوبتی}}
درصورتی که حرکت به نقطه‌ای چندنوبتی باشد، کوتاهترین مسیر ممکن را یافته و نوبت‌به‌نوبت شروع به حرکت برروی آن میکند. درصورتی که طی این مدت، حرکت به آن نقطه غیرممکن شود (بعنوان‌مثال به دریا رسیده و توانایی عبور از آنرا نداشته باشد و یا واحد دیگری زودتر از آن در مقصد ساکن شود)، باید بایستد و منتظر دستورات جدید باشد. لازم به ذکر است که در هر لحظه میتوانید با دادن دستور حرکت جدید به یک واحد محرک و یا لغو دستور قبلی، آن را نگه دارید و یا مقصد آن را تغییر دهید.

\subsubsection*{{\titr امتیاز حرکت (MP) }}
\addcontentsline{toc}{subsubsection}{{\fehrestContent امتیاز حرکت (MP) }}
کلیه واحد‌های قابل حرکت دارای تعداد محدودی پوینت هستند که میتوانند درهرنوبت برای حرکت مصرف کنند. هرگاه امتیازهای حرکت یک واحد تمام شود، تا نوبت بعد نمیتواند حرکتی انجام دهد.

\subsubsection*{{\titr قوانین مصرف MP}}
\addcontentsline{toc}{subsubsection}{{\fehrestContent قوانین مصرف MP}}
هزینه MP برای حرکت بین هر دو خانه، تنها براساس خانه‌ی مقصد محاسبه میشود و بستگی به پستی و بلندی زمین آن دارد. بعنوان مثال زمین‌های پست چمنی و دشت‌ها 1 پوینت و جنگل‌ها ۲ پوینت نیاز دارند. علاوه بر این برای عبور از رودخانه‌ها کل MP مصرف میشود (مگر آنکه برروی آن جاده‌ای باشد).
درنظر داشته باشید که یک واحد تنها درصورتی نمیتواند حرکت کند که MPای نداشته باشد؛ بعبارت‌دیگر هرمقداری MP داشته باشد (مستقل از هزینه ورود به خانه) میتواند به آن وارد شود و سپس درصورت تمام شدن پوینتهایش همانجا می‌ایستد.

\subsubsection*{{\titr جاده‌ و راه‌آهن‌ها}}
\addcontentsline{toc}{subsubsection}{{\fehrestContent جاده‌ و راه‌آهن‌ها}}جاده‌ها و راه‌آهن‌های موجود در نواحی دوستانه و بیطرف، هزینه MP را به‌شدت کاهش میدهند. تا زمانی که واحد محرک MP دارد و از خانه‌ای حاوی جاده/راه‌آهن به خانه دیگری حاوی جاده/راه‌آهن حرکت کند تنها کسری از هزینه MP حرکت عادی را می‌پردازد.
برای عبور از جاده/راه‌آهن‌های ساخته شده برروی رودخانه‌ها، باید هزینه‌ی MP اضافه‌ای بپردازیدکه این هزینه باید به مقدار کسر شده اضافه شود و بعبارتی کامل پرداخته شود. درصورتی که تکنولوژی ساخت‌و‌ساز را داشته باشید، نیازی به پرداخت این پنالتی نیست.

\subsubsection*{{\titr محدودیت‌های انباشتن}}
\addcontentsline{toc}{subsubsection}{{\fehrestContent محدودیت‌های انباشتن}}
\begin{itemize}
\item در هر نوبت، تنها یک واحد نظامی میتواند در یک نقطه از نقشه ساکن شود.
\item در هر نوبت، تنها یک واحد غیرنظامی میتواند در یک نقطه از نقشه ساکن شود.
\item در هر نوبت، تنها یک واحد نظامی و یک واحد غیرنظامی میتوانند در یک نقطه از نقشه ساکن شوند.
\item درصورت پر بودن یک ناحیه، تنها درصورتی یک واحد اضافه‌ای میتواند وارد آن ناحیه شود که MP لازم برای عبور و درناحیه خروج از آن ناحیه را داشته باشد.
\end{itemize}

\subsubsection*{{\titr حرکت در جنگ}}
\addcontentsline{toc}{subsubsection}{{\fehrestContent حرکت در جنگ}}
برای دستور حمله کافیست به واحد‌های نظامی خود دستور حرکت به خانه‌ای را بدهید که واحد‌های نظامی دشمن در آنها مستقرند.
واحد‌های نظامی اطراف خود دارای ناحیه‌ای بنام ناحیه کنترل (ZOC) هستند؛ درصورتی که یک واحد بین دو خانه موجود در ناحیه‌کنترل دشمن حرکت کند، کلیه MP خود را از دست میدهد. توجه کنید در صورتی که این به یک واحد غیرنظامی دستور بدهید که به خانه‌ای با واحد نظامی دشمن برود، این دستور نباید با موفقیت انجام شود.

% Crow
\subsection*{{\titr Combat}}
\addcontentsline{toc}{subsection}{{\fehrestContent Combat}}
نبرد‌ها بین دوتمدن شکل می‌گیرند که درگیر جنگ با یکدیگر هستند. 
\subsubsection*{{\titr اعلان جنگ}}
\addcontentsline{toc}{subsubsection}{{\fehrestContent اعلان جنگ}}
برای اعلان جنگ روش‌های متفاوتی وجود دارد و همچنین ممکن است شما از دشمن خود نامه اعلان جنگ دریافت کنید.
\subsubsection*{{\titr اعلان جنگ دیپلماتیک}}
\addcontentsline{toc}{subsubsection}{{\fehrestContent اعلان جنگ دیپلماتیک}}
شما می‌توانید از طریق پنل دیپلماسی، بر علیه یکی از تمدن‌ها اعلان جنگ کنید. همچنین برای اعلان جنگ علیه ایالت‌های شهری، می‌توانید روی شهر مورد نظر کلیک کنید و گزینه "اعلان جنگ" را انتخاب کنید
\subsubsection*{{\titr حمله به واحد‌های دیگر}}
\addcontentsline{toc}{subsubsection}{{\fehrestContent حمله به واحد‌های دیگر}}
شما می‌توانید به سادگی، به یکی از واحد‌های خود دستور بدهید تا به واحد‌های یک تمدن دیگر حمله کنند. در صورتی که در آن لحظه، در حال جنگ با تمدن مورد نظر در نباشید، از شما درخواست می‌شود تا اعلان جنگ را تایید کنید. در صورت تایید، حمله آغاز می‌شود و در صورت عدم تایید، دستور حمله لغو خواهد شد.
\subsubsection*{{\titr یگان‌هایی که می‌توانند بجنگند}}
\addcontentsline{toc}{subsubsection}{{\fehrestContent یگان‌هایی که می‌توانند بجنگند}}
هر نوع یگان جنگی می‌تواند با یگان جنگی دشمن بجنگد. یگان‌های غیرنظامی مثل کارگرها یا مهاجرین نمی‌توانند شروع‌کننده‌ی نبرد باشند. زمانی هم که به آن‌ها حمله می‌شود به گروگان گرفته می‌شوند و به کارگر تبدیل می‌شوند. 
\subsubsection*{{\titr نبرد با سلاح سرد}}
\addcontentsline{toc}{subsubsection}{{\fehrestContent نبرد با سلاح سرد}}
این نوع نبرد زمانی رخ می‌دهد که یک یگان با سلاح سرد (هر یگانی که قدرت حمله از راه دور ندارد) به یک واحد یا شهر دشمن حمله می‌کند. در این نوع نبرد، از سلامت هر یگان به اندازه‌ی قدرت یگان مقابلش کم می‌شود. همچنین توجه کنید که شرایط اقلیمی و دفاعی یگان‌ها هم در این قدرت در نظر گرفته می‌شوند. توجه کنید که همه‌ی یگان‌ها در سالم‌ترین حالت خود به اندازه‌ی ۱۰ امتیاز ضربه سلامتی دارند.\\
\textbf{نتیجه‌ی نبرد:}
در نتیجه‌ی نبرد یگان‌ها آسیب می‌بینند و از سلامتی آن‌ها کاسته می‌شود. در صورتی که سلامتی یک یگان به صفر برسد، آن یگان به طور کلی از بین می‌رود. در این صورت، اگر یگان مقابل هنوز از بین نرفته باشد، وارد کاشی این یگان می‌شود. با ورود این یگان نظامی به کاشی، یگان‌های غیرنظامی دشمن به گروگان گرفته می‌شوند. در غیر این‌صورت یگان‌ها در سر جای خود باقی می‌مانند و تنها از سلامت آن‌ها کاسته می‌شود.
علاوه بر این، در طی حمله، \lr{MP}های یک یگان مصرف می‌شوند. همچنین، پس از نبرد تمام یگان‌های دخیل در نبرد تجربه (XP) دریافت می‌کنند.
\subsubsection*{{\titr نبرد از راه دور}}
\addcontentsline{toc}{subsubsection}{{\fehrestContent نبرد از راه دور}}
برخی از واحدها مانند تیراندازان، منجنیق‌هامی‌تواند واحد دیگر را ببیند که چیزی، دید بین واحد را مسدود کرده باشد (مانند کوه، تپه یا جنگل). یک واحد می‌تواند واحد‌هایی که درونشان زمین‌های مسدود کننده دارند را ببینند اما واحد‌هایی که پشت آن ها هستند را نمی‌تواند ببیند.
\\\noindent \textbf{قدرت در نبردهای راه دور:}هر  واحدی که می‌تواند در نبرد از راه دور شرکت کند یک مقدار «قدرت نبرد راه دور» دارد که در نبرد برای تعیین برنده، این عدد با قدرت نبرد حریف مقایسه می‌شود.
\\\noindent \textbf{دامنه:}هر واحد یک مقدار تخت  عنوان «دامنه پرتاب» داره که نشان دهنده فاصله‌‌ای است که می‌تواند هدف را مورد اصابت قرار دهد. به عنوان مثال دامنه ۲ به این معناست که آن واحد می‌تواند، واحد‌هایی که یک یا دو واحد فاصله دارند را مورد اصابت قرار دهد.
\\\noindent \textbf{نتیجه نبرد از راه دور:} در انتهای نبرد از راه دور، واحد‌های حریف ممکن است بس آسیب باقی مانده باشند، کمی آسیب دیده باشند و یا به طور کلی نابود شده باشند. واحد‌های مهاجم هیچگونه آسیبی را متحمل نمی‌شوند. در صورت نابودی هدف، واحد مهاجم به صورت خودکار، به خانه خالی منتقل نمی‌شود اما شما می‌توانید یک واحد را با امتیازات حرکتی، به خانه خالی شده منتقل کنید. واحد‌های مهاجم و مدافع، می‌توانند در تنیجۀ نبرد، امتیاز تجربه (xp) کسب کنند
\subsubsection*{{\titr جوایز نبرد}}
\addcontentsline{toc}{subsubsection}{{\fehrestContent جوایز نبرد}}
واحد‌ها در طول نبرد می‌توانند جوایز مختلفی کسب کنند. از موقعیت مکانی آن واحد و وضعیت دفاعی گرفته تا شرایط خاص دیگر. برخی از این جوایز فقط روی واحد‌های هجومی اعمال می‌شود و برخی روی واحد‌های تدافعی و برخی روی هر دو. رایج‌ترین این جوایز از زمین یا ناحیه‌ای است که یک واحد اشغال می‌کند یا هنگامی که یک واحد دفاعی، مستحکم شده است.
\subsubsection*{{\titr پاداش زمین}}
\addcontentsline{toc}{subsubsection}{{\fehrestContent پاداش زمین}}
واحد‌های دفاعی از اشغال نواحی جنگل، تپه یا کوهستان، امتیاز پاداش مهمی کسب می‌کنند. واحد‌های هجومی سرد در صورت حمله به یک حریف در سوی دیگر یک رودخانه، با جریمه همراه خواهند شد. واحد‌های هجومی در صورت حمله از نواحی تپه‌ای، امتیاز پاداش خواهند گرفت.
\subsubsection*{{\titr استحکامات}}
\addcontentsline{toc}{subsubsection}{{\fehrestContent استحکامات}}
خیلی از واحد‌ها قابلیت مستحکم شدن دارند. به این معنا که آن واحد می‌تواند در موقعیت کنونی خود کار‌های دفاعی بسازد. این کار به آن واحد، امتیازات دفاعی خوبی می‌دهد که آن را در برابر نابودی مقاوم‌تر می‌کند. استحکامات همیشه دفاعی هستند و در صورتی که آن واحد تکان بخورد و یا حمله کند، استحکامات از بین می‌روند.
\\یک واحد مستحکم شده، غیرفعال است تا زمانی که شما به صورت دستی آن را فعال کنید.
\\\noindent \textbf{کدام واحد‌ها می‌توانند مستحکم شوند؟} اکثر واحد‌های سرد و دوربرد می‌توانند مستحکم شوند. واحد‌های غیر نظامی، سواره، زرهی نمی‌توانند مستحکم شوند. این واحد‌ها می‌توانند در حالت خواب باشند که یعنی غیرفعالند تا زمانی که به آنها حمله شود و یا شما به صورت دستی آنها را فعال کنید اما امتیازات دفاعی دریافت نخواهند کرد.
\\\noindent \textbf{پاداش امتیازات استحکام:} میزان پاداش امتیاز بستگی به طول مدتی دارد که یک واحد مستحکم شده است. یک واحد، در دور اولی که مستحکم شده است،25 درصد پاداش امتیاز می‌گیرد و در دور‌های بعدی 50 درصد پاداش امتیاز می‌گیرد.
\\\noindent \textbf{دستور هشدار:} دستور هشدار، مانند دستور استحکام است، با این تفاوت که آن واحد، در صورتی که یک واحد دشمن را در نزدیکی خود ببیند، بیدار می‌شود. واحد بیدار شده، مادامی که تکان نخورد و یا حمله نکند می‌تواند پاداش امتیاز استحکام را حفظ کند.
\subsubsection*{{\titr نبرد شهری}}
\addcontentsline{toc}{subsubsection}{{\fehrestContent نبرد شهری}}
شهر‌ها بزرگ‌اند و اهداف مهمی برای دشمنان هستند و در صورتی که مستحکم شده باشند و توسط واحد‌هایی از آنها دفاع شود، تصرف آنها کار دشواری است که در صورت موفقیت، جوایز و غنائم بسیار زیادی را به ارمغان می‌آورد. در واقع تنها راهی که یک تمدن را به صورت کلی از بازی خارج کنید، این است که تمامی شهر‌های آن را تصرف کنید. اگر این کار را با تعداد کافی از حریفان خود انجام دهید،  پیروزی شکوهمندانه‌ای نصیب شما خواهد شد.
\\\noindent \textbf{قدرت در نبردهای شهری:} شهر‌ها نیز مانند واحد‌ها، قدرت نبرد دارند قدرت هر شهر، به اندازه شهر، موقعیت آن ( شهر‌های روی تپه قدرت بیشتری دارند ) و اینکه صاحب شهر دور آن دیوار و یا سازه‌های دفاعی ساخته باشد، بستگی دارد.\\
قدرت یک شهر، نمایانگر قدرت نبرد و قدرت نبرد دوربرد آن است. در هنگام نبرد، به خاطر حملات دشمن، امتیاز ضربه شهر ممکن است کاهش پیدا کند اما قدرت نبرد و قدرت نبرد دوربرد شهر، در هر شرایطی ثابت است و ربطی به میزان آسیبی که شهر دیده ندارد.
\\\noindent \textbf{امتیاز ضربه شهر:} یک شهر کاملا سالم، 20 امتیاز ضربه دارد. در صورتی که آسیب ببیند، این امتیاز، کاهش پیدا خواهد کرد و در صورتی که این امتیاز، صفر شود، واحد دشمن می‌تواند وارد شهر شود و ناحیه مربوطه را تصرف کند.
\subsubsection*{{\titr هجوم به شهر‌ها با واحد‌های دوربرد}}
\addcontentsline{toc}{subsubsection}{{\fehrestContent هجوم به شهر‌ها با واحد‌های دوربرد}}
برای هدف قرار دادن یک شهر توسط واحد‌های دوربرد، ابتدا باید آن واحد را طوری تکان دهید که، شهر هدف در دامنه پرتاب واحد قرار گیرد. بسته به میزان قدرت هجوم، امتیاز ضربه شهر کاهش پیدا می‌کند.در تهاجم، واحد‌ها آسیبی نخواهند دید. توجه داشته باشید که تهاجم دوربرد، امتیاز ضربه شهر را نمی‌تواند زیر 1 بیاورد و تصرف شهر، حتما باید توسط واحد‌های سرد انجام شود.
\subsubsection*{{\titr هجوم به شهر‌ها با واحد‌های سرد}}
\addcontentsline{toc}{subsubsection}{{\fehrestContent هجوم به شهر‌ها با واحد‌های سرد}}
هنگامی که یک واحد در نبرد سرد با یک شهر قرار می‌گیرد می‌تواند امتیاز ضربه آن را کاهش دهد و خود واحد نیز ممکن است آسیب ببیند. یک شهر همیشه با تمام قدرت نبرد خود، دفاع می‌کند و قدرت نبرد آن ربطی یه امتیاز ضربه آن نخواهد داشت.
\subsubsection*{{\titr واحد‌های پادگان در شهر‌ها}}
\addcontentsline{toc}{subsubsection}{{\fehrestContent واحد‌های پادگان در شهر‌ها}}
صاحب یک شهر ممکن است یک پادگان نظامی در شهر، به منظور تقویت قدرت دفاعی شهر بسازد. قدرت واحد‌های پادگان، به قدرت شهر افزوده می‌شود. واحد پادگان در طول نبرد آسیبی نمی‌بیند، اما در صورتی که شهر تصرف شود، این واحد‌ها نابود خواهند شد\\
یک واحد مستقر در شهر، می‌تواند به واحد‌های دشمن که شهر را احاطه کرده‌اند حمله کند، اما در آن صورت پاداش امتیاز پادگان را از دست خواهد داد و همچنین اگر نبرد سرد باشد، ممکن است آسیب هم ببیند.
\subsubsection*{{\titr حمله شهر‌ها به نیرو‌های مهاجم}}
\addcontentsline{toc}{subsubsection}{{\fehrestContent حمله شهر‌ها به نیرو‌های مهاجم}}
هر شهر یک قدرت نبرد دوربرد دارد که در ابتدای نبرد، با قدرت کل شهر برابر است و دامنۀ 2 دارد و می‌تواند هر واحد دشمن که در این فاصله باشد را هدف قرار دهد. این قدرت در صورت آسیب دیدن شهر، کم نمی‌شود و همیشه برابر با همان قدرت اول است تا زمانی که شهر تصرف شود.
\subsubsection*{{\titr ترمیم آسیب‌های شهر}}
\addcontentsline{toc}{subsubsection}{{\fehrestContent ترمیم آسیب‌های شهر}}
یک شهر در هر نوبت از بازی، یک امتیاز ضربه‌اش جبران ترمیم می‌شود، حتی اگر در حال نبرد باشد. بنابراین برای تصرف یک شهر، مهاجم باید در هر نوبت، بیش از یک امتیاز ضربه، به آن شهر آسیب بزند.
\subsubsection*{{\titr تصرف شهر‌ها}}
\addcontentsline{toc}{subsubsection}{{\fehrestContent تصرف شهر‌ها}}
هنگامی که امتیاز ضربه یک شهر، به صفر می‌رسد، واحد دشمن می‌تواند وارد شهر شود (فارغ از واحد‌هایی که از قبل در شهر بوده‌اند) و شهر را تصرف کند. در این شرایط، شهر به قلمرو مهاجم اضافه می‌شود.
\subsubsection*{{\titr سلاح‌های محاصره‌ای}}
\addcontentsline{toc}{subsubsection}{{\fehrestContent سلاح‌های محاصره‌ای}}
برخی از سلاح‌های دوربرد، تحت عنوان سلاح‌های محاصره‌ای دسته بندی می‌شوند. منجنیق، بالیستا، تربوشه، و غیره در این دسته هستند. این واحد‌ها، در صورت حمله به شهر‌های دشمن، امتیازات پاداش می‌گیرند. این واجد‌ها در مقابل حملات سرد، بسیار آسیب پذیرند و برای دفع حمله دشمن، باید حتما همراه واحد‌های سلاح سرد باشند. برای استفاده از این نوع نیروها ابتدا باید در حالت حمله قرار بگیرند که 1 نوبت(turn) طول می‌کشد و سپس در نوبت بعدی می‌توانند حمله کنند.
\subsubsection*{{\titr آسیب‌های جنگ}}
\addcontentsline{toc}{subsubsection}{{\fehrestContent آسیب‌های جنگ}}
یک واحد کاملا سالم، 10 امتیاز ضربه دارد و در هنگام نبرد، این امتیاز کاهش پیدا می‌کند. اگر این امتیاز به صفر برسد، آن واحد نابود می‌شود. یک واحد آسیب دیده، ضعیف‌تر از یک واحد سالم است و در هر لحظه ممکن است نابود شود. بنابراین این ایده خوبی است که در مواقعی که ممکن است، این واحد‌ها را با واحد‌های سالم جا‌به‌جا کنید و به واحد‌های آسیب دیده، فرصت ترمیم بدهید.
\subsubsection*{{\titr اثرات آسیب}}
\addcontentsline{toc}{subsubsection}{{\fehrestContent اثرات آسیب}}
یک واحد آسیب دیده، نسبت به یک واحد کاملا سالم، اثرگذاری کمتری در نبرد دارد. هر چه یک واحد آسیب بیشتری دیده باشد، هجوم آن واحد نیز، آسیب کمتری به واحد‌های دشمن وارد می‌کند. از درصد آسیبی که هجوم یک واحد، به واحد‌های دشمن می‌زند به میزان نصف درصد امتیاز ضربه‌ای که واحد مهاجم از دست داده، کم می‌شود. اگر یک واحد مهاجم، 8 امتیاز ضربه داشته باشد (20 درصد از دست داده)، آسیبی که به واحد دشمن می‌زند، 90 درصد خواهد بود زیرا نصف 20 درصد از تاثیرگذاری آن کم شده است.
\subsubsection*{{\titr ترمیم آسیب}}
\addcontentsline{toc}{subsubsection}{{\fehrestContent ترمیم آسیب}}
برای ترمیم آسیب، یک واحد باید به اندازه یک نوبت، غیرفعال باشد. میزان ترمیم بستگی به موقعیت واحد دارد. اگر در شهر باشد، به ازای هر دور، 3 امتیاز ضربه ترمیم می‌شود. در قلمرو دوستانه 2 امتیاز ضربه و در قلمرو دشمن یا بی طرف، 1 امتیاز ضربه به ازای هر دور غیرفعال بودن، ترمیم می‌شود. برخی از ارتقا‌ها باعث می‌شود سرعت ترمیم واحد‌ها افزایش پیدا کند.
\subsubsection*{{\titr دکمه "استحکام تا زمان ترمیم"}}
\addcontentsline{toc}{subsubsection}{{\fehrestContent دکمه "استحکام تا زمان ترمیم"}}
اگر یک واحد، آسیب دیده باشد، در صورت انتخاب آن، آن واحد مستحکم می‌شود و تا زمانی که کامل ترمیم شود، در موقعیت خود باقی می‌ماند

\subsection*{{\titr Ruins}}
\addcontentsline{toc}{subsection}{{\fehrestContent Ruins}}
در طی بازی شما با کاشی هایی به اسم Ruins مواجه خواهید شد. این کاشی های بازمانده civilization های قبلی هستند که قبل از آغاز بازی شما به وجود و نابود شده اند. اگر شما این کاشی ها را کشف و هر نوع نیرویی را بر روی آن ها قبل از بازیکن های دیگر ببرید به شما قدرت هایی می دهد که در ادامه درباره آنها بحث می کنیم. همچنین پس از ورود شما به این کاشی ها آن ها از بین می‌روند.
\begin{enumerate}
	\item تکنولوژی های مجانی
	\item برداشتن \lr{fog of war} از برخی از کاشی‌ها
	\item اضافه کردن جمعیت (1 واحد)
	\item جعبه های طلا
	\item کارگر و settler
\end{enumerate}

\subsection*{{\titr Cities}}
\addcontentsline{toc}{subsection}{{\fehrestContent Cities}}
شهرها برای موفقیت تمدن شما حیاتی هستند. آنها واحدها، ساختمان ها می سازند. آنها به شما اجازه می دهند در مورد فناوری های جدید تحقیق کنید و ثروت جمع آوری کنید. بدون شهرهای قدرتمند و دارای موقعیت خوب نمی توانید پیروز شوید.
\\\noindent \textbf{چگونه شهر‌ها را بسازیم؟}
شهرها توسط واحدهای settler یا همان مهاجر ساخته می شوند. مهاجرها می‌توانند در موقعیت کنونی خود نسبت به ایجاد یک شهر جدید اقدام کنند.
\\\noindent \textbf{کجا شهرها را بسازیم؟}
شهرها باید در مکان هایی با مواد غذایی و تولید فراوان و با دسترسی به منابع ساخته شوند. اغلب ایده خوبی است که یک شهر را بر روی یک رودخانه یا قسمت ساحلی بسازید. شهرهایی که بر روی تپه‌ها ساخته شده‌اند، امتیازات دفاعی به دست می‌آورند، که تسخیر آنها را برای دشمنان سخت‌تر می‌کند.
\\\noindent \textbf{بنر شهر:}
شامل وضعیت تولید (طلا، غذا و…) شهر، جمعیت آن  و کاشی‌ها، ساختمان‌ها و محل کار شهرونداند شهر است. علاوه بر این شامل نام شهر و قدرت رزمی شهر است. همچنین از این صفحه قابلیت رفتن به شهرهای دیگر قلمرو فراهم است. به جز این‌ها نوبت‌های باقی‌مانده تا دستیابی به تکنولوژی، ساختمان، یگان، و شهروند جدید را نشان می‌دهد.
\\\noindent \textbf{"قفل کردن" یک شهروند به یک کاشی:}
شما می‌توانید، به یک شهروند سفارش دهید که در یک کاشی خاص (کار نشده) کار کند. اگر یک شهروند بیکار در دسترس باشد، آن شهروند می‌رود تا در آن کاشی کار کند. در غیر این صورت، نیاز است تا یک شهروند را از یک خانه بی‌کار کنید و بعد این اقدام را انجام دهید.
\\\noindent \textbf{حذف یک شهروند از کار:}
با حذف یک شهروند از کار، شهروند کار آن کاشی را متوقف می‌کند و بیکار می شود و در لیست شهروند بیکار ظاهر می‌شود. سپس می توانید به آن شهروند دستور دهید تا در کاشی دیگری کار کند.
\\\noindent \textbf{خروجی شهر:}
این تابلو نشان می‌دهد که شهر چقدر مواد غذایی، تولید، طلا، علم تولید می‌کند. همچنین نشان می دهد که تا زمانی که مرز شهر افزایش یابد و تا زمانی که جمعیت شهر افزایش یابد، چند دور می‌گذرد.
\\\noindent \textbf{خلاصه خروجی \lr{civilization}:}
این خط از داده‌ها موارد زیر را نشان می‌دهد:
\begin{itemize}
	\item تمدن شما هر نوبت از این شهر چقدر علم به دست می‌آورد
	\item شهر شما چقدر طلا دارد و چقدر درآمد دارد
	\item شادی تمدن شما و پیشرفت شما به سوی عصر طلایی بعدی
	\item منابع استراتژیک شهری شما
\end{itemize}
\noindent \textbf{شهروندان بیکار:} این بخش فقط در صورت داشتن شهروندان بیکار قابل مشاهده است.

شهروندان این لیست نه متخصص هستند و نه در زمین های اطراف شهر خود کار می‌کنند: آنها بیکار هستند. یک شهروند بیکار در هر نوبت فقط 1 محصول تولید می‌کند، در حالی که همچنان همان مقدار غذا را مصرف می‌کند که همه شهروندان دیگر مصرف می‌کنند.
\\\noindent \textbf{کاشی بخر:}
این به شما این امکان را می‌دهد تا زمانی که می توانید بهای کاشی را بپردازید، یک کاشی بخرید. توجه کنید که کاشی خریداری شده، مجاور یکی از کاشی‌های شهر است.
\\\noindent \textbf{خرید:}
شهر کالایی را که در حال حاضر روی آن کار می کند خریداری نمی‌کند. پس از خرید، شهر به ساخت مورد ادامه خواهد داد (مگر اینکه نتواند این کار را انجام دهد).  در واقع خرید مستقل از ساختن است. به عنوان مثال، اگر شهری در حال کار بر روی یک کماندار است و 4 دور باقی مانده است و شما یک کماندار خریداری می‌کنید، کماندار خریداری شده را فوراً دریافت می‌کنید و آن که در حال ساخت است 4 دور بعد دریافت می‌کنید- مگر اینکه پس از خرید اولین کماندار، تولید را تغییر دهید.

\subsubsection*{{\titr واحدها در شهرها}}
\addcontentsline{toc}{subsubsection}{{\fehrestContent واحدها در شهرها}}
\begin{itemize}
	\item \textbf{واحدهای رزمی:} تنها یک واحد رزمی ممکن است در یک زمان یک شهر را اشغال کند. گفته می‌شود که آن واحد نظامی شهر را Garrison می‌کند و امتیاز دفاعی قابل توجهی (به اندازه‌ی یک‌سوم قدرت یگان) به شهر اضافه می‌کند. واحدهای رزمی اضافی ممکن است در شهر حرکت کنند، اما نمی‌توانند نوبت خود را در آنجا به پایان برسانند.
	\item \textbf{واحدهای غیر رزمی:} فقط یک واحد غیر جنگی (کارگر، ساکن یا شخص بزرگ) ممکن است در یک زمان یک شهر را اشغال کند. دیگران می توانند حرکت کنند، اما نمی توانند حرکت خود را در شهر پایان دهند. بنابراین، یک شهر ممکن است در انتهای یک دور حداکثر دو واحد در خود داشته باشد: یک واحد رزمی و یک واحد غیر رزمی.
\end{itemize}

\subsubsection*{{\titr ساخت و ساز در شهرها}}
\addcontentsline{toc}{subsubsection}{{\fehrestContent ساخت و ساز در شهرها}}
شما ممکن است ساختمان‌ها، یا واحدهایی را در یک شهر بسازید ولی فقط یک مورد را می‌توان در یک زمان ساخت. پس از اتمام ساخت، پیام هشدار "انتخاب تولید" ظاهر می‌شود. و لازم است قبل از اتمام نوبت، ساخت‌وساز بعدی انتخاب شود.
\\\noindent \textbf{منوی ساخت شهر:}
منوی ساخت شهر تمام واحدها، ساختمان‌ها و را که در آن زمان می‌توانید در آن شهر بسازید را نمایش می‌دهد. با افزایش فناوری شما موارد جدید ظاهر می‌شوند و موارد منسوخ ناپدید می‌شوند. هر مدخل به شما می گوید که چند نوبت طول می‌کشد تا ساخت و ساز کامل شود.
\\\noindent \textbf{تغییر ساخت و ساز:}
اگر می‌خواهید آنچه را که شهر در حال ساخت است را تغییر دهید باید در صفحه ی شهر اینکار را انجام دهید. تولیدی که قبلا برای آیتم اصلی هزینه شده است برای آیتم جدید اعمال نمی‌شود و از بین می‌رود.
\\\noindent \textbf{ساخت واحدها:}
شما می‌تونید هر تعداد واحد در شهر بسازید (تا زمانی که منابع ضروری را در اختیار دارید). از آنجایی که شما فقط می‌توانید یک واحد رزمی و یک واحد غیر رزمی در شهر داشته باشید، احتمالا مجبور میشوید تا واحد نو ساخت را بعد از ساخت به بیرون شهر منتقل کنید.
\\\noindent \textbf{ساخت ساختمان‌ها:}
فقط یک ساختمان از هر نوع در شهر می‌تواند ساخته شود. شما نمی‌توانید ساختمان های مشابه و کپی شده در یک شهر داشته باشید. زمانی که شما ساختمان را ساختید آن ساختمان از منوی ساخت شهر ناپدید می‌گردد(البته شما میتوانید همان ساختمان را در شهر دیگری بسازید).
\\\noindent \textbf{کار کردن در زمین:}
شهرها بر اساس زمین های اطرافشان رشد می‌کنند. شهروندان آنها بر روی زمین کار می‌کنند و غذا، ثروت، تولید و علم را از کاشی‌ها برداشت می‌کنند. شهروندان می‌توانند کاشی هایی را کار کنند که در فاصله دو کاشی از شهر و در محدوده تمدن آنها قرار دارند. فقط یک شهر می تواند یک کاشی واحد را کار کند، حتی اگر در فاصله دو کاشی از بیش از یک کاشی باشد.
\\\noindent \textbf{بهبود زمین:}
در حالی که تیکه­ زمین­‌های خاص به طور طبیعی مقادیر خوبی از غذا، ثروت و غیره را فراهم می‌کنند، بسیاری از تیکه­ زمین‌­ها را می‌توان بهبود داد تا حتی بیشتر فراهم کرد، بنابراین رشد، ثروت، بهره‌وری یا علم شهر را افزایش داد. برای بهبود زمین‌ها باید «کارگران» بسازید. هنگامی که یک کارگر دارید، می‌توانید به او دستور دهید تا پیشرفت‌هایی را بسازد - مانند مزارع، معادن، مدارس و غیره - که زمین‌های اطراف شهرهای شما را بسیار پربارتر می‌کند.
\subsubsection*{{\titr نبرد شهری}}
\addcontentsline{toc}{subsubsection}{{\fehrestContent نبرد شهری}}
شهرها ممکن است مورد حمله و تصرف واحدهای دشمن قرار گیرند. هر شهر دارای یک آمار "قدرت رزمی \lr{(Combat Strength)}" است که با توجه به موقعیت شهر، وسعت آن، اینکه آیا واحدهای نظامی در آن شهر مستقر (garrisoned) هستند یا خیر، و اینکه آیا ساختمان های دفاعی مانند دیوارها در شهر ساخته شده اند یا خیر مشخص می شود. هر چه قدر قدرت دفاعی یک شهر بیشتر تصرف شهر دشوارتر است مگر اینکه شهر بسیار ضعیف باشد یا واحد مهاجم بسیار قوی باشد، برای تصرف یک شهر چندین واحد، چندین نوبت نیاز دارند. برای جزئیات بیشتر در مورد جنگ به طور کلی به "نبرد" مراجعه کنید.

\subsubsection*{{\titr حمله کردن به یک شهر}}
\addcontentsline{toc}{subsubsection}{{\fehrestContent حمله کردن به یک شهر}}
برای حمله به یک شهر دشمن، به واحد‌های  سلاح سرد خود دستور دهید وارد hex شهر شوند. یک راند نبرد به جریان می‌­افتد و هم واحد­ها و هم شهر ممکن است آسیب ببینند. اگر \lr{points hit} نیرو­های شما به صفر برسد، از بین می روند. اگر \lr{hit points} های شهر به صفر برسد، نیروهای شما شهر را تصرف می‌کند.


\subsubsection*{{\titr حمله کردن با نیروهای دوربرد}}
\addcontentsline{toc}{subsubsection}{{\fehrestContent حمله کردن با نیروهای دوربرد}}
اگرچه شما می‌توانید به یک شهر حمله کنید و آن را با یگان های دوربرد و آن را فرسوده کنید، شما نمی‌توانید شهر را با نیروهای دوربرد تصرف کنید، باید یک \lr{melee unit} را به داخل شهر منتقل کنید تا آن را تصرف کنید.

\subsubsection*{{\titr دفاع از یک شهر}}
\addcontentsline{toc}{subsubsection}{{\fehrestContent دفاع از یک شهر}}
تعدادی از کارها وجود دارد که می توانید برای بهبود سیستم دفاعی شهر انجام دهید. شما ممکن است یک واحد قوی را در شهر "مستقر \lr{(garrison)}" کنید.\\
\\
همچنین می‌توانید دیوارها و قلعه‌هایی بسازید که استحکام شهر را بهبود می بخشد. شهری روی تپه یک جایزه دفاعی نیز دریافت می‌کند. مهم نیست که یک شهر چقدر قدرتمند است، اما داشتن واحدهایی در خارج از شهر که از آن پشتیبانی می‌کنند، آسیب رساندن به واحدهای مهاجم و جلوگیری از احاطه کردن شهر و دریافت پاداش های بزرگ جانبی در برابر آن بسیار مهم است.\\

\noindent \textbf{فتح یک شهر:}
وقتی نیروی شما وارد شهر یک دشمن می‌شود، دو انتخاب پیش روی شماست: می‌توانید شهر را نابود کنید یا می‌توانید آن را ضمیمه و یا بخشی از امپراتوری خود کنید. هرکدام سود‌ها و هزینه‌های خود را دارد. این تصمیم باید در لحظه‌ی اشغال شهر گرفته شود.

\noindent \textbf{نابود کردن شهر:}
اگر یک شهر را نابود کنید، آن شهر دیگر از بین رفته است. همچنین تمام ساختمان‌ها و شهروندانش نیز دیگر وجود نخواهند داشت. با این وجود چندین دلیل خوب برای نابود کردن یک شهر وجود دارد، که اکثرا برای خشنود سازی مردم است.

\subsubsection*{{\titr شهر های نابود ناپذیر}}
\addcontentsline{toc}{subsubsection}{{\fehrestContent شهر های نابود ناپذیر}}
شما نمی‌توانید شهر هایی را که تاسیس کرده‌اید را نابود کنید. (بعضی از دیگر تمدن‌ها می‌‌توانند ولی شما خیر.) همچنین، شما نمی‌توانید یک شهر خودمختار یا پایتخت یک تمدن را نابود کنید.

\noindent \textbf{ضمیمه کردن شهر:} اگر یک شهر را ضمیمه کنید، آن را به بخش از امپراتوری خود تبدیل می‌کنید. شما کنترل مطلق بر آن شهر خواهید داشت، انگار مثل اینکه آن شهر را خود بنا کرده‌اید. یک جنبه منفذ برای ضمیمه کردن یک شهر این است که ضمیمه کردن، شهروندهای شما را ناراضی می‌کند،  و شما نیاز پیدا خواهید کرد که ساختمان‌های شادآور بسازید مانند دادگاه و استادیوم ورزشی  و یا آن ها را به منابع لاکچری وصل کنید تا با ناخشنودی مفرط مقابله شود. ضمیمه کردن شهر های زیاد به طور مداوم ممکن است باعث پیشرفت نکردن امپراتوری شما شود.


\subsection*{{\titr Buildings}}
\addcontentsline{toc}{subsection}{{\fehrestContent Buildings}}

\subsubsection*{{\titr ساختمان‌ها}}
\addcontentsline{toc}{subsubsection}{{\fehrestContent ساختمان‌ها}}
یک شهر چیزی بیشتر از تعدادی کاشی است. یک شهر شامل مدارس، کتابخانه‌ها، بازارها، بانک‌ها و پادگان‌ها است. ساختمان‌ها نشان دهنده پیشرفت‌ها و ارتقا‌هایی هستند که شما در یک شهر انجام می‌دهید. 
ساختمان‌ها می‌توانند نرخ رشد شهر ها را افزایش دهند و تولید را سرعت بخشند، علم یک شهر را افزایش دهند، قدرت دفاعی آن را بهبود بخشند و خیلی کار های مفید دیگر انجام دهند.
شهری که ساختمان نداشته باشد‌، بسیار ضعیف خواهد بود و احتمالا نسبتا کوچک خواهد ماند درحالیکه شهری با ساختمان های بسیار، میتواند رشد کند و بر جهان تسلط یابد.


\subsubsection*{{\titr نحوه ساخت ساختمان‌ها}}
\addcontentsline{toc}{subsubsection}{{\fehrestContent نحوه ساخت ساختمان‌ها}}
وقتی شهری آماده ساخت چیزی باشد، منو تولید شهر باید ظاهر شود و در دسترس باشد. اگر ساختمانی برای ساخت در دسترس باشد، باید در این منو مشاهده شود. در این منو باید قادر باشید تا دستور دهید که شهر شروع به ساخت و ساز کند.


\subsubsection*{{\titr تغییر ساخت و ساز و خرید ساختمان‌ها}}
\addcontentsline{toc}{subsubsection}{{\fehrestContent تغییر ساخت و ساز و خرید ساختمان‌ها}}
شما امکان تغییر سفارش های ساخت و ساز در صفحه شهر را دارید. برای مثال می توانید طلا خرج کنید و ساختمانی را خریداری کنید.


\subsubsection*{{\titr پیشنیاز‌های ساختمان‌ها}}
\addcontentsline{toc}{subsubsection}{{\fehrestContent پیشنیاز‌های ساختمان‌ها}}
به استثنای بناهای تاریخی که هیچ پیش نیازی ندارند و در دسترس هستند، برای ساخت در ابتدای بازی، شما نیاز به دانش یک تکنولوژی خاص برای ساخت دارید. برای مثال، برای ساختن پادگان، شما باید کار با برنز را یاد بگیرید.
برخی از ساختمان‌ها، پیش نیاز های منابع نیز دارند. برای مثال، یک شهر برای ساختن سیرک، باید منبع بهبودیافته‌ای از اسب یا عاج در نزدیکی خود داشته باشد.
همچنین بعضی از ساختمان‌ها دارای پیش نیاز ساختمانی هستند. برای مثال برای ساختن یک معبد در شهر، شما باید قبلا یک بنای تاریخی در آنجا ساخته باشید.

\subsubsection*{{\titr تعمیر و نگهداری ساختمان‌ها}}
\addcontentsline{toc}{subsubsection}{{\fehrestContent تعمیر و نگهداری ساختمان‌ها}}
ساختمان ها یک نقطه ضعف دارند و آن هم نگهداری آنها می‌باشد. برای نگهداری آنها باید طلا بپردازید. بسته به ساختمان موردنظر‌، قیمت میتواند از 1 تا 10 در هر نوبت متغیر باشد و از خزانه شما کسر می‌شود.


\subsubsection*{{\titr کاخ}}
\addcontentsline{toc}{subsubsection}{{\fehrestContent کاخ}}
کاخ یک ساختمان خاص است. کاخ بصورت خودکار در اولین شهری که می سازید ، ظاهر میشود و آن شهر را به پایتخت امپراطوری شما تبدیل میکند. اگر پایتخت شما تصرف شود ، کاخ شما بصورت خودکار در شهر دیگری بازسازی می شود و آن شهر را بعنوان پایتخت جدید شما قرار می دهد.اگر بعدا پایتخت اصلی خود را پس بگیرید ، کاخ به شهر اصلی خود برمیگردد.
کاخ ، مقدار کمی از تولید ، علم ، طلا را برای تمدن شما فراهم میکند. 


\subsubsection*{{\titr شهرهای تسخیر شده}}
\addcontentsline{toc}{subsubsection}{{\fehrestContent شهرهای تسخیر شده}}
اگر شهری تسخیر شود، ساختمان‌های آن نیز تسخیر می‌شود. ساختمان های نظامی شهر(معابد، سربازخانه و … ) هم تخریب می‌شوند. تمامی ساختمان های دیگر 66 درصد احتمال دارد که دست نخورده گرفته شود.

\subsection*{{\titr Food}}
\addcontentsline{toc}{subsection}{{\fehrestContent Food}}
غذای فراوان مهم‌ترین عامل رشد تمدن انسانی‌ست. تا زمانی که انسان‌ها مجبور بودند تک‌تک لحظه‌های بیداری خود را به شکار و پیدا کردن غذا برای خود و خانواده‌شان می‌گذراندند، زمان و انرژی اندکی برای جستجو در شگفتی‌های جهان داشتند. با وجود غذای اضافه، همه چیز ممکن ممکن می‌شود.

\subsubsection*{{\titr شهرها و غذا
}}
\addcontentsline{toc}{subsubsection}{{\fehrestContent شهرها و غذا
}}
هر شهر به دو واحد غذا برای هر شهروند در هر نوبت نیاز دارد تا شهروندانش از گرسنگی نمیرند. شهرها با به کار گماشتن شهروندان روی کاشی‌های اطراف شهر غذا به دست می‌آورند. هر شهر می‌تواند روی هر کاشی‌ای که با شهر حداکثر دو واحد فاصله دارد و در حوزه‌ی مرزهای تمدن است، کار کند. البته این امر به این شرط است که شهر دیگری از تمدن روی آن کاشی مشغول به کار نباشد.
توجه کنید که در صورتی که غذای کافی برای شهر فراهم نباشد، شهروندان آن از گرسنگی می‌میرند و شهر تا زمانی که نتواند غذای شهروندان خود را فراهم کند، آن‌ها را از دست می‌دهد.
لازم است امکان مشخص کردن کاشی‌هایی که شهر روی آن‌ها کار می‌کند فراهم شود.

\subsubsection*{{\titr تهیه‌ی غذای بیشتر}}
\addcontentsline{toc}{subsubsection}{{\fehrestContent تهیه‌ی غذای بیشتر}}
برخی کاشی‌ها به طور پیشفرض غذای تولیدی در هر کاشی می‌تواند با کاشی‌های دیگر متفاوت باشد. علاوه بر این، کارگران با ساخت برخی پیشرفت‌ها غذای تولیدی در یک کاشی را افزایش دهند.\\
\\
\noindent \textbf{منابع امتیازی:}
کاشی‌هایی که منبع امتیازی دارند، در صورتی که پیشرفت متناسب آن‌ها روی آن کاشی انجام شود، غذای خیلی زیادی تولید می‌کنند.

\noindent \textbf{ویژگی‌های زمین:}
با توجه به ویژگی‌های زمین، کاشی‌ها مقدار غذای متفاوتی ارائه می‌دهند.

\noindent \textbf{پیشرفت‌ها:}
کارگارها با ساختن پیشرفت از نوع مزرعه روی کاشی‌ها می‌توانند باعث افزایش غذای تولیدی آن کاشی شوند.

\noindent \textbf{ساختمان‌ها:}
برخی از ساختمان‌ها بر روی غذای تولیدی شهر تاثیر می‌گذارند.

\noindent \textbf{نارضایتی شهروندان:}
در صورتی که مردم از تمدن شما نارضایتی داشته باشند، غذای اضافی تولید‌شده توسط شهرها ۶۷درصد کاهش پیدا می‌کند.

\subsubsection*{{\titr رشد شهر}}
\addcontentsline{toc}{subsubsection}{{\fehrestContent رشد شهر}}
در هر نوبت، شهروندان غذای اضافی خود را ذخیره می‌کنند. در صورتی که این غذا به حد مشخصی برسد، غذای اضافی مصرف شده، و یک شهروند جدید به شهر اضافه می‌شود. این حد به صورت نمایی با اضافه شدن شهروندان جدید افزایش پیدا می‌کند. وضعیت غذای تولیدی شهر، غذای ذخیره‌شده و غذای موردنیاز برای تولید شهروند بعدی از طریق پنل لیست شهرها در دسترس است.

\subsubsection*{{\titr تولید غذا و \lr{settler}ها}}
\addcontentsline{toc}{subsubsection}{{\fehrestContent تولید غذا و \lr{settler}ها}}
واحدهای settler تنها در شهرهایی با حداقل دو شهروند ساخته می‌شوند. در زمان تولید این واحدها، غذای اضافی شهر توسط settlerها مصرف می‌شود (دور ریخته می‌شود). یعنی تا زمان تولید settler غذای اضافه‌ای به ذخیره‌ی غذای شهر اضافه نمی‌شود.



\subsection*{{\titr Technology}}
\addcontentsline{toc}{subsection}{{\fehrestContent Technology}}


\subsubsection*{{\titr تکنولوژی}}
\addcontentsline{toc}{subsubsection}{{\fehrestContent تکنولوژی}}
پیشرفت تکنولوژی در یک تمدن موجب قدرتمندتر شدن، بزرگ‌تر شدن و باهوش‌تر شدن آن می‌شود. برای اینکه یک تمدن بتواند با تمدن‌های دیگر رقابت کند، همواره نیاز دارد تا دانش خود نسبت به تکنولوژی‌های مختلف را افزایش دهد.

\subsubsection*{{\titr تکنولوژی و جام‌ها}}
\addcontentsline{toc}{subsubsection}{{\fehrestContent تکنولوژی و جام‌ها}}
با دستیابی به یک تکنولوژی، می‌توانید به برخی یگان‌ها، ساختمان‌ها و منابع جدید دسترسی پیدا کنید که باعث پیشرفت تمدن می‌شود. برای دستیابی به یک تکنولوژی، باید از «جام‌ها» (که نمایش‌دهنده میزان دانشی است که تمدن دارد) استفاده کنید.
بعد از گذشت هر نوبت، مقداری جام برحسب دانش تمدن دریافت می‌شود. هر تکنولوژی نیاز به مقدار جام خاصی دارد. پس از اینکه مطالعه یک تکنولوژی را شروع کنید، تمامی جام موجود صرف آن می‌شود و باید برای تکنولوژی بعدی مجدداً جام جمع‌آوری کنید.

\subsubsection*{{\titr نحوه جمع‌آوری جام}}
\addcontentsline{toc}{subsubsection}{{\fehrestContent نحوه جمع‌آوری جام}}
جامی که دریافت می‌کنید به جمعیت شهرهای شما بستگی دارد. در هر نوبت، 3 جام برای پایتخت و 1 جام برای هر شهروند دریافت می‌کنید. مثلاً اگر تنها یک شهر با جمعیت 5 داشته باشید، در هر نوبت 8 جام دریافت می‌کنید. هرچه جام بیشتری داشته باشید، سرعت مطالعه و دستیابی به یک تکنولوژی بیشتر خواهد شد.

% TODO: fix this
\subsubsection*{{\titr افزایش مقدار جام}}
\addcontentsline{toc}{subsubsection}{{\fehrestContent افزایش مقدار جام}}
 با انجام کارهای زیر، می‌توانید سرعت دریافت جام را افزایش دهید.

\subsubsection*{{\titr خرابه‌ها}}
\addcontentsline{toc}{subsubsection}{{\fehrestContent خرابه‌ها}}
برخی از خرابه‌ها ممکن است که یک تکنولوژی در اختیار شما قرار دهد.

\subsubsection*{{\titr معامله}}
\addcontentsline{toc}{subsubsection}{{\fehrestContent معامله}}
پس از این که به تکنولوژی نوشتن دست یافتید، می‌توانید با یک تمدن دیگر قرارداد تحقیقاتی ببندید. در این قرارداد، طرفین معامله 150 سکه پرداخت می‌کنند و در ازای آن، 15 درصد به سرعت مطالعه تکنولوژی‌های جدید افزایش می‌یابد.

\subsubsection*{{\titr ساختمان‌های علمی}}
\addcontentsline{toc}{subsubsection}{{\fehrestContent ساختمان‌های علمی}}
ساختن برخی از ساختمان‌ها (مانند کتابخانه یا دانشگاه) می‌تواند باعث افزایش سرعت دریافت جام شود. مثلاً ساخت یک کتابخانه موجب افزایش 50 درصدی جام دریافتی حاصل از شهروندان می‌شود.

\subsubsection*{{\titr انتخاب تکنولوژی برای مطالعه}}
\addcontentsline{toc}{subsubsection}{{\fehrestContent انتخاب تکنولوژی برای مطالعه}}
به محض اینکه اولین شهر (پایتخت) خود را بنا کنید، می‌توانید از منوی انتخاب تکنولوژی یک تکنولوژی را برای مطالعه انتخاب کنید. پس از اینکه مطالعه تکنولوژی پایان یافت و به آن تکنولوژی دست یافتید، می‌توانید تکنولوژی دیگری را برای مطالعه انتخاب کنید.

\subsubsection*{{\titr منوی انتخاب تکنولوژی}}
\addcontentsline{toc}{subsubsection}{{\fehrestContent منوی انتخاب تکنولوژی}}
در منوی انتخاب تکنولوژی، باید مطالعه پایان‌یافته را نمایش دهید. بعد از آن نیز باید تمام تکنولوژی‌های ممکن برای مطالعه را نمایش دهید و برای هرکدام مدت زمان لازم برای دستیابی به آن تکنولوژی، و همچنین ساختمان‌ها و ویژگی‌هایی که آن تکنولوژی در اختیار تمدن قرار می‌دهد را مشخص کنید. همچنین در این منو می‌توانید درخت تکنولوژی را مشاهده کنید.

\subsubsection*{{\titr تغییر مطالعه}}
\addcontentsline{toc}{subsubsection}{{\fehrestContent تغییر مطالعه}}
شما می‌توانید با ورود به منوی انتخاب تکنولوژی، تکنولوژی مورد مطالعه را تغییر دهید. پس از تغییر دادن مطالعه، دانش حاصل از مطالعه قبلی از بین \underline{نمی‌رود} و می‌توانید هر زمانی مجدداً مطالعه آن تکنولوژی را ادامه دهید.

\subsubsection*{{\titr تکنولوژی‌های قابل مطالعه}}
\addcontentsline{toc}{subsubsection}{{\fehrestContent تکنولوژی‌های قابل مطالعه}}
در ابتدای بازی معمولاً چند تکنولوژی ابتدایی (مانند دامداری، تیراندازی، سفال‌کاری و معدن‌کاری) قابل مطالعه هستند. بقیه تکنولوژی‌ها برای مطالعه یک یا چند پیش‌نیاز دارند. پس از اینکه تکنولوژی‌های پیش‌نیاز مطالعه شدند، می‌توانید این تکنولوژی‌ها را در منوی انتخاب تکنولوژی مشاهده و در صورت تمایل مطالعه کنید. برای مثال تکنولوژی ریاضیات، نیاز به تکنولوژی چرخ دارد و تکنولوژی چرخ نیاز به تکنولوژی تیراندازی و دامداری دارد. بنابراین تا زمانی که تکنولوژی‌های تیراندازی و دامداری مطالعه نشوند، مطالعه تکنولوژی چرخ و ریاضیات ممکن نیست.



\subsection*{{\titr Workers}}
\addcontentsline{toc}{subsection}{{\fehrestContent Workers}}


\subsubsection*{{\titr کارگرها و پیشرفت}}
\addcontentsline{toc}{subsubsection}{{\fehrestContent کارگرها و پیشرفت}}
کارگرها مردان و زنانی هستند که امپراطوری شما را می‌سازند، جنگل‌ها را پاکسازی می‌کنند و مزرعه‌هایی می‌سازند که منبع تغذیه شهر شماست.
آنها معدن‌هایی حفر می‌کنند که برای شما طلا و آهن فراهم می‌کند، جاده‌‌هایی می‌سازند که شهرهای شما را به هم متصل می‌کند.
با این که کارگرها واحد نظامی نیستند نقش به سزایی در امپراطوری شما دارند.
پیشرفت‌هایی که به کمک کارگرها حاصل می‌شود شامل افزایش تولید، طلا و غذا می‌شود. اگر شما تمدن خود را ارتقا ندهید، توسط تمدن‌های ارتقا یافته دیگر نابود خواهید شد.

\subsubsection*{{\titr ساختن کارگرها}}
\addcontentsline{toc}{subsubsection}{{\fehrestContent ساختن کارگرها}}
کارگرها مانند واحدهای دیگر در شهرها ساخته می‌شوند.

\subsubsection*{{\titr کارگران در مبارزه}}
\addcontentsline{toc}{subsubsection}{{\fehrestContent کارگران در مبارزه}}
کارگران واحدهای غیر نظامی هستند. هنگامی که یک واحد دشمن وارد محدوده آنها می شود، تسخیر می‌شوند.
آنها می‌توانند توسط حملات دامنه دار نیز آسیب ببینند (آنها مانند واحدهای دیگر بهبود می یابند، اما اینطور نیست که ارتقا یابند). کارگران نمی‌توانند به هیچ واحد دیگری حمله کنند یا به آن آسیب برسانند. 

\subsubsection*{{\titr ساختن جاده‌ها}}
\addcontentsline{toc}{subsubsection}{{\fehrestContent ساختن جاده‌ها}}
هنگامی که تمدن آنها به فناوری Wheel دست یافت، کارگران می‌توانند جاده بسازند.
جاده‌ها را می‌توان در قلمرو دوستانه، بی طرف یا دشمن ساخت. آنها را می توان در هر زمینی ساخت و بر روی هر ویژگی، به جز کوه‌ها، یخ و نواحی (tile) آبی ساخت.

\noindent \textbf{زمان لازم برای ساختن یک جاده:} 
برای  هر کارگر ۳ دور (turn) زمان لازم است تا یک جاده را در یک ناحیه (tile) بسازد.

\subsubsection*{{\titr جایی که می توان پیشرفت ها را ایجاد کرد}}
\addcontentsline{toc}{subsubsection}{{\fehrestContent جایی که می توان پیشرفت ها را ایجاد کرد}}
فقط در مکان های مناسب می‌توان پیشرفت ها را ساخت (برای مثال مزارع نمی‌توانند روی یخ ساخته شوند).
به طور کلی، مزارع را می‌توان در هر ناحیه (tile) که حاوی منبعی نباشد، ساخت. اگر ناحیه (tile) حاوی یک منبع است، فقط می‌توان پیشرفت مناسب آن ناحیه را ایجاد کرد.

\subsubsection*{{\titr ترک و از سرگیری یک پروژه پیشرفت}}
\addcontentsline{toc}{subsubsection}{{\fehrestContent ترک و از سرگیری یک پروژه پیشرفت}}
اگر پروژه ای را در وسط رها کنید و بعداً همان پروژه را از سر بگیرید، زمانی که قبلا صرف ساخت پروژه کرده بودید ذخیره می‌شود. با این حال، اگر پروژه ها را تغییر دهید، تمام پیشرفت پروژه قبلی از بین می رود.

\subsubsection*{{\titr پیشرفت مزرعه}}
\addcontentsline{toc}{subsubsection}{{\fehrestContent پیشرفت مزرعه}}
مزرعه اولین و رایج ترین پیشرفت ساخته شده است. همه تمدن‌ها با دانستن نحوه کشاورزی شروع می شوند. پیشرفت مزرعه را می‌توان در اکثر نواحی (tile) ساخت.\\
فناوری مورد نیاز: کشاورزی (در ابتدای بازی به دست آمد)\\
در چه جاهایی ممکن است ساخته شود: هر جایی جز یخ.\\
مدت زمان ساخت: 6 نوبت
\begin{itemize}
	\item \textbf{:Forest}    مزارع ممکن است در \lr{forest tiles} پس از یادگیری \lr{mining tech} ساخته شوند. وقتی مزرعه ساخته می‌شود، forest حذف می‌شود.\\
کل زمان ساخت: 10 نوبت

	\item \textbf{:Jungle} مزارع ممکن است در \lr{jungle tiles} پس از یادگیری \lr{bronze working tech} ساخته شوند. وقتی مزرعه ساخته می شود، jungle حذف می شود.\\
کل زمان ساخت: 13 نوبت

	\item \textbf{:Marsh} مزارع ممکن است در \lr{marsh tiles} پس از یادگیری \lr{masonry tech} ساخته شوند. وقتی مزرعه ساخته می شود، marsh حذف می شود.\\
کل زمان ساخت: 12 نوبت

	\item \textbf{منابع قابل دسترسی:} مزارع می توانند به منابع گندم دسترسی داشته باشند و  خروجی \lr{tile}ها را ۱واحد غذا و ۱واحد طلا افزایش دهند.

\end{itemize}



\subsubsection*{{\titr پیشرفت معدن}}
\addcontentsline{toc}{subsubsection}{{\fehrestContent پیشرفت معدن}}
پیشرفت معدن زمانی حاصل می‌شود که تمدن شما فناوری معدن را به دست آورد. پیشرفت معدن برای افزایش بازده تولید بسیاری از tileها استفاده می شود و همچنین دسترسی به منابع مختلفی را باز می کند. معدن به اندازه کشاورزی اهمیت دارد.\\
تکنولوژی مورد نیاز: معدن\\
در چه جاهایی ممکن است ساخته شود: معادن را فقط می توان بر روی  \lr{tiles Hills} یا Resource ساخت. معادن خروجی یک tile را ۱واحد افزایش می‌دهد.\\
مدت زمان ساخت: 6 نوبت
\begin{itemize}
	\item \textbf{:Forest}    معادن ممکن است در \lr{forest tiles} ساخته شوند. وقتی معدن ساخته می شود، forest حذف می‌شود.\\
کل زمان ساخت: 10 نوبت

	\item \textbf{:Jungle} معادن ممکن است در \lr{jungle tiles} پس از یادگیری \lr{bronze working tech} ساخته شوند. وقتی معدن ساخته می‌شود، jungle حذف می‌شود.\\
کل زمان ساخت: 13 نوبت

	\item \textbf{:Marsh} معادن ممکن است در \lr{marsh tiles} پس از یادگیری \lr{masonry tech} ساخته شوند. وقتی معدن ساخته می‌شود، marsh حذف می‌شود.\\
کل زمان ساخت: 12 نوبت


	\item \textbf{منابع قابل دسترسی:} معادن دسترسی به آهن، زغال سنگ، آلومینیوم، اورانیوم، جواهرات و طلا را باز می کنند.
\end{itemize}

\subsubsection*{{\titr غارت راه‌ها و پیشرفت‌ها}}
\addcontentsline{toc}{subsubsection}{{\fehrestContent غارت راه‌ها و پیشرفت‌ها}}
واحدهای دشمن می‌توانند جاده‌ها و پیشرفت‌ها را «غارت» کنند و آنها را موقتاً بی‌فایده کنند (بدون منبع، بدون پاداش حرکت، و غیره. گویی کارگر هرگز آن پیشرفت یا  راه را نساخته است). یک واحد حتی ممکن است پیشرفت تمدن خود را غارت کند (معمولاً هنگامی که یک شهر در شرف غارت شدن توسط دشمن است).

\subsubsection*{{\titr تعمیر راه‌ها و پیشرفت‌ها}}
\addcontentsline{toc}{subsubsection}{{\fehrestContent تعمیر راه‌ها و پیشرفت‌ها}}
یک کارگر ممکن است یک جاده غارت‌شده یا پیشرفت غارت‌شده را تعمیر کند. برای تعمیر هر یک جاده یا پیشرفت،
کارگر باید به مدت ۳ نوبت (turn) کار کند.

\subsubsection*{{\titr لیست اقدامات کارگرها}}
\addcontentsline{toc}{subsubsection}{{\fehrestContent لیست اقدامات کارگرها}}
\noindent \textbf{ساختن جاده:} می‌توانید بر روی ناحیه‌ (tile) فعلی یک جاده بسازید. جاده ها را می توان روی هر ناحیه قابل عبور ساخت.

\noindent \textbf{ساختن راه‌آهن(مسیر ریلی):} بر روی ناحیه‌ (tile) فعلی یک راه‌آهن بسازید. راه‌آهن را می توان روی هر ناحیه قابل عبور ساخت.

\noindent \textbf{ساختن مرزعه:} یک مزرعه بر روی ناحیه‌ (tile) فعلی ایجاد کنید. مزارع، تولید مواد غذایی یک ناحیه را افزایش می دهند. برخی از منابع مانند گندم نیاز دارند تا بر روی مزارع ساخته شوند تا مورد استفاده قرار گیرند.

\noindent \textbf{ساختن معدن:} بر روی ناحیه‌ (tile) فعلی یک معدن ایجاد کنید و تولید آن را بهبود بخشید. برخی از منابع مانند آهن یا سنگهای قیمتی برای مورد استفاده قرار گرفتن به معادن نیاز دارند.

\noindent \textbf{ساختن پست تجاری \lr{(trading post)}:} یک پست تجاری در ناحیه‌ (tile) فعلی ایجاد کنید. این پست، خروجی طلای ناحیه‌ (tile) را بهبود می بخشد.

\noindent \textbf{ساختن کارخانه چوب‌بری:} روی ناحیه‌ (tile) فعلی یک کارخانه چوب‌بری بسازید که باید با ویژگی Forest پوشیده شود. کارخانه های چوب Production اضافی تولید می‌کنند.

\noindent \textbf{ساختن چراگاه:} یک چراگاه روی ناحیه‌ (tile) انتخاب شده بسازید. منابعی مانند اسب و گاوها به مراتع نیاز دارند.

\noindent \textbf{ساختن کمپ:} بر روی ناحیه‌ (tile) انتخاب شده یک کمپ بسازید تا از منابعی مانند خز و گوزن استفاده کنید.

\noindent \textbf{ساختن \lr{Plantation}:} روی ناحیه‌ (tile) فعلی یک Plantation بسازید. Plantation به منظور استفاده از بسیاری از منابع لوکس مورد نیاز است.

\noindent \textbf{ساختن منبع \lr{(quarry)}:} بر روی ناحیه‌ (tile) فعلی یک منبع بسازید. منابع سنگ مرمر نیاز دارد به منبع نیاز دارند تا بر روی آنها ساخته شوند و مورد استفاده قرار گیرند.

\noindent \textbf{حذف \lr{jungle}:} به کارگر دستور می‌دهید ناحیه‌ (tile) انتخاب شده را از هر ویژگی که jungle دارد، پاک کند. این کار همه مزایای ارائه شده توسط jungle را از بین می‌برد.

\noindent \textbf{حذف \lr{forest}:} به کارگر دستور می دهید ناحیه‌ (tile) انتخاب شده را از هر ویژگی که forest دارد، پاک کند. این کار همه مزایای ارائه شده توسط forest را از بین می‌برد.

\noindent \textbf{حذف \lr{marsh}:} به کارگر دستور می دهید ناحیه‌ (tile) انتخاب شده را از هر ویژگی که marsh دارد، پاک کند.

\noindent \textbf{حذف مسیر:} به کارگر دستور می‌دهید ناحیه‌ (tile) انتخاب شده را از هرگونه مسیر (railroads یا road)، پاک کند.

\noindent \textbf{تعمیر:} ر گونه آسیب ناشی از غارت ناحیه‌ (tile) را تعمیر کنید. تا زمانی که این کاشی تعمیر نشود، نمی توان از منابع و پیشرفت موجود در آن استفاده کرد.



\subsection*{{\titr طلا}}
\addcontentsline{toc}{subsection}{{\fehrestContent طلا}}
با طلا می‌توان ارتش جدید ساخت، ساختمان و عمارت بنا کرد، دوستی با یک منطقه را بدست آورد و حتّی تهدید تمدن رقیب را با دادن هدیه به فرصت تبدیل کرد.
\subsubsection*{{\titr چه جاهایی طلا جمع کنیم؟}}
\addcontentsline{toc}{subsubsection}{{\fehrestContent چه جاهایی طلا جمع کنیم؟}}
طلا از راه‌های متنوعی بدست می‌آید. کار روی کاشی‌های اطراف شهر مهم‌ترین منبع طلای شما خواهد بود. منابع دیگری هم وجود دارند که در ادامه می‌آید.

انواع زمین‌هایی که کار روی آن‌ها طلای مورد نیاز شما را تأمین می‌کند:
\begin{itemize}
	\item کاشی‌های ساحل
	\item کاشی‌های رودخانه
	\item واحه‌ها
\end{itemize}
\noindent \textbf{منابع:} همه منابع، (علی الخصوص خود طلا!) شامل طلا هستند.
\\\noindent \textbf{ساختمان‌ها:} اکثر ساختمان‌ها از جمله فروشگاه‌ها و بانک‌ها درآمد طلا را افزایش می‌دهند. به خصوص اگر بازرگان‌های متخصص را مسئول آن‌ها قرار دهید.
\\\noindent \textbf{خرابه‌های باستانی:} با کشف این خرابه‌ها، شاید طلا دریافت کنید.
\\\noindent \textbf{فتح شهر:} به تصرف درآوردن یک شهر یا تمدن می‌تواند طلا برای شما تأمین کند.
\\\noindent \textbf{دیپلماسی:} هنگام مذاکره با دیگر تمدن‌ها ممکن است طلا دریافت کنید.
\subsubsection*{{\titr راه‌های خرج کردن طلا}}
\addcontentsline{toc}{subsubsection}{{\fehrestContent راه‌های خرج کردن طلا}}
\noindent \textbf{نگهداری یگان‌ها و ساختمان‌ها:} هر یگان و ساختمانی هزینه نگهداری دارد که باید هر نوبت پرداخت شود. (لازم به ذکر است این هزینه‌ها ثابت نیست و نسبت به سطح سختی بازی متفاوت خواهد بود.)
\\\noindent \textbf{نگهداری جاده‌ها:} برای هر جاده‌ای که می‌سازید، مقداری طلا هزینه می‌شود. اگر جاده‌های تمدن‌های دیگر را هم به منطقه تحت سلطه خود اضافه کنید، برای آن هم باید هزینه نگهداری پرداخت کنید.
\\\noindent \textbf{خرید کاشی:} می‌توانید منطقه تحت سلطه خود را با خرید کاشی‌های جدید گسترش دهید.
\\\noindent \textbf{خرید یگان، ساختمان یا عمارت:} می‌توانید طلای خود را صرف خرید یگان‌ها، ساختمان‌ها یا عمارت‌های جدید کنید.
\\\noindent \textbf{ارتقا دادن یگان‌های موجود:} با گذشت زمان، به تکنولوژی‌های جدیدی دست پیدا می‌کنید که به شما این امکان را می‌دهد تا نیروهای قوی‌تری بسازید. وقتی این اتفاق افتاد، می‌توانید یگان‌های قدیمی را هم با صرف طلا قدرتمندتر کنید. (برای مثال وقتی به تکنولوژی آهن دست پیدا کردید، می‌توانید هر یگان جنگجویی را به شمشیرزن ارتقا دهید.)
یگان مورد نظر باید در منطقه تحت سلطه شما باشد تا ارتقا صورت گیرد.
\\\noindent \textbf{دیپلماسی:} دلایل متعددی برای داد‌ و ستد طلا بین شما و تمدن دیگر می‌تواند وجود داشته باشد. از جمله برقراری صلح یا اجیر کردن یک تمدن برای حمله به تمدن دیگر. طلا برای مذاکره بسیار حیاتی است.

دو راه برای داد و ستد طلا وجود دارد:
\begin{itemize}
	\item همه یک جا
	\item قسطی در هر نوبت (برای مثال 5 عدد طی 30 نوبت). در این روش اگر دو تمدن با هم وارد جنگ شوند، قرارداد منحل می‌شود.
\end{itemize}
\noindent \textbf{از دست دادن شهر:} اگر یک شهر یا تمدن، شهر شما را تصرف کند، مقداری از طلایتان را از دست می‌دهید. (بدیهی است خود شهر هم دیگر برای شما نخواهد بود)
\subsubsection*{{\titr تمام شدن طلا}}
\addcontentsline{toc}{subsubsection}{{\fehrestContent تمام شدن طلا}}
اگر طلای شما به صفر برسد، در صورت نیاز به کسر شدن طلا، این مقدار از دانش شما کم می‌شود. پس در این شرایط حتماً سعی کنید طلا بدست آورید، زیرا به صورت جدی پیشرفت شما در تکنولوژی کند می‌شود و نسبت به تمدن‌های دیگر شکننده خواهید شد.


\subsection*{{\titr شادی}}
\addcontentsline{toc}{subsection}{{\fehrestContent شادی}}
شادی معیاری برای سنجش رضایت شهروندان شماست. به عنوان یک قاعده، هرچه کل جمعیت شما بیشتر باشد، همه ناراضی تر می‌شوند. جمعیت ناراضی خیلی سریع رشد نمی‌کند، و یک جمعیت بسیار ناراضی بر کیفیت جنگ ارتش های شما نیز تأثیر می‌گذارد. شادی تمدن شما در نوار وضعیت\lr{(status bar)} صفحه اصلی (در سمت چپ بالا) نمایش داده می‌شود. با دقت آن را زیر نظر داشته باشید. اگر به صفر برسد، جمعیت شما بی قرار می‌شود. اگر شروع به پایین رفتن به اعداد منفی کرد، با مشکل مواجه می‌شوید. (در ضمن، می توانید با نگه داشتن نشانگر موس روی این عدد، یک تصویر لحظه‌ای از جمعیت خود دریافت کنید.)

\subsubsection*{{\titr شادی ابتدایی}}
\addcontentsline{toc}{subsubsection}{{\fehrestContent شادی ابتدایی}}
میزان شادی که تمدن شما با آن شروع می‌شود، بر اساس شرایط دشواری بازی تعیین می‌شود. لحظه ای که اولین شهر خود را می سازید، این مقدار شروع به کاهش می‌کند.
\subsubsection*{{\titr مواردی که باعث نارضایتی می‌شود}}
\addcontentsline{toc}{subsubsection}{{\fehrestContent مواردی که باعث نارضایتی می شود}}
\begin{itemize}
	\item \textbf{جمعیت خام:} با رشد تمدن شما، مردم به طور افزاینده ناراضی شده  و چیزهای بیشتری برای سرگرم کردن خود تقاضا می‌کنند.
	\item \textbf{تعداد شهرها:} با افزایش تعداد شهرها در تمدن شما، نارضایتی شما نیز افزایش می‌یابد. به عبارت دیگر، تمدنی با 2 شهر با جمعیت 1، ناراضی تر از تمدنی با 1 شهر با جمعیت 2 است. هرچند که هر دو دارای جمعیت کلی یکسانی می‌باشند.
	\item \textbf{شهرهای ضمیمه شده:} اگر شهرهای خارجی را تصرف و ضمیمه کنید، جمعیت شما آن را دوست نخواهد داشت.
\end{itemize}
\subsubsection*{{\titr مواردی که باعث شادی می‌شود}}
\addcontentsline{toc}{subsubsection}{{\fehrestContent مواردی که باعث شادی می‌شود}}
\begin{itemize}
	\item \textbf{منابع لوکس:} منابع را در قلمرو خود بهبود بخشید یا برای آنها با سایر تمدن ها تجارت کنید. هر نوع منابعی شادی جمعیت شما را بهبود می بخشد (اما خوشحالی اضافی برای داشتن چندین نسخه از یک لوکس(Luxuries) یکتا دریافت نمی‌کنید).
	\item \textbf{ساختمان ها:} ساختمان های خاص شادی جمعیت شما را افزایش می دهند. اینها عبارتند از استادیوم‌ها، سیرک، تئاتر، و غیره. هر ساختمان در هر نقطه از تمدن شما ساخته شود شادی کلی شما را افزایش می‌دهد. (بنابراین دو استادیوم دو برابر بیشتر شادی ایجاد می کنند نسبت به یکی، برخلاف لوکس‌ها).
	\item \textbf{فناوری‌ها \lr{(technologies)}:} فناوری‌ها به خودی خود باعث شادی نمی‌شوند، اما قفل ساختمان‌ها، شگفتی‌ها، منابع و سیاست‌های اجتماعی را باز می‌کنند.
\end{itemize}
\subsubsection*{{\titr نارضایتی}}
\addcontentsline{toc}{subsubsection}{{\fehrestContent نارضایتی}}
\begin{itemize}
	\item \textbf{ناراضی(Unhappy):} وقتی شادی شما منفی است، جمعیت شما "ناراضی" هستند. اگر جمعیت شما ناراضی است، رشد شهرهای شما به کلی متوقف می‌شود، نمی‌توانید هیچ مهاجری(Settlers) بسازید، و قدرت واحدهای نظامی شما ۲۵ درصد کاهش پیدا می‌کند.
\end{itemize}
به یاد داشته باشید که نارضایتی دائمی نیست. شما همیشه می‌توانید از طریق روش های ذکر شده در بالا، شادی شهروندان خود را افزایش دهید.
\subsection*{{\titr برد و باخت}}
\addcontentsline{toc}{subsection}{{\fehrestContent برد و باخت}}
راه‌های متفاوتی برای برد در این بازی وجود دارد. برای مثال می‌توانید با رسیدن به بالاترین درجه‌ی تکنولوژی پیروز بازی شوید. می‌توانید با برتری فرهنگ یا زیرکی در سیاست، دیگر تمدن‌ها را در هم بشکنید. یا می‌توانید با نابود کردن تمام تمدن‌های دیگر پیروز شوید. هر تمدنی که زودتر به یکی از شرایط پیروزی در بازی برسد پیروز نهایی بازی خواهد بود.
\subsubsection*{{\titr چطور ببازیم؟}}
\addcontentsline{toc}{subsubsection}{{\fehrestContent چطور ببازیم؟}}
سه روش برای باختن وجود دارد:
\begin{enumerate}
	\item برد یک تمدن دیگر: با رسیدن یک تمدن به یکی از شروط پیروزی، تمام تمدن‌ها دیگر مغلوب خواهند شد.
	\item از دست دادن آخرین شهر: اگر شما تمام شهرهای خود را از دست بدهید بازنده خواهید بود.
	\item رسیدن سال ۲۰۵۰: اگر با فرارسیدن سال ۲۰۵۰ هنوز هیچ تمدنی پیروز نشده باشد، بازی به طور خودکار به پایان خواهد رسید و تمدنی که بیشترین امتیاز را داشته باشد پیروز بازی خواهد بود. (توضیحات بیشتر در ادامه)
\end{enumerate}
\subsubsection*{{\titr چطور پیروز شویم؟}}
\addcontentsline{toc}{subsubsection}{{\fehrestContent چطور پیروز شویم؟}}
دو روش برای برد در بازی وجود دارد:
\begin{enumerate}
	\item \textbf{چیرگی:} اگر آخرین تمدنی باشید که همچنان صاحب پایتخت اصلی‌اش است، پیروز خواهید شد. یعنی اگر شما تمام پایتخت‌ها دیگر را فتح کنید پیروز خواهید شد. البته این می‌تواند غلط‌انداز باشد. برای مثال اگر شما در یک بازی پنج‌نفره پایتخت‌های سه تمدن دیگر را فتح کنید و تمدن دیگر پایتخت اصلی شما را فتح کند او برنده‌ی بازی خواهد بود. یعنی مهم نیست شما پایتخت چند تمدن را فتح کنید. در هر صورت آخرین تمدنی که صاحب پایتخت اصلی خود باشد پیروز خواهد شد. بنابراین اگر شما پایتخت اصلی خود را از دست داده باشید قبل از پس گرفتن آن امکان پیروزی از راه چیرگی را نخواهید داشت. البته همچنان می‌توانید از راه‌های دیگر پیروز شوید.
	\begin{itemize}
		\item از بین بردن پایتخت‌ها: از بین بردن پایتخت‌ها در بازی ممکن نیست. هیچ روشی برای این کار وجود ندارد. در بدترین حالت می‌توانید جمعیت یک پایتخت را به ۱ برسانید. دقت کنید که شما می‌توانید با گرفتن تمام شهر‌های یک تمدن آن تمدن را از بین ببرید ولی پایتختش در بازی باقی خواهد ماند.
		\item پایتخت اصلی و پایتخت فعلی: اگر شما پایتخت اصلی خود را از دست بدهید، یک شهر دیگر به طور خودکار به عنوان پایتخت جدید انتخاب خواهد شد. این شهر تمام ویژگی‌های پایتخت اصلی را دارد به  جز دو مورد: اول اینکه در محاسبات پیروزی از راه چیرگی، پایتخت اصلی شما ملاک خواهد بود. دوم اینکه این شهر برخلاف پایتخت‌های اصلی تمدن‌ها می‌تواند کاملا از بین برود. در صورت بازپس‌گیری پایتخت اصلی، به طور خودکار پایتخت شما به پایتخت اصلیتان تغییر خواهد کرد.
	\end{itemize}
	\item \textbf{پایان زمان:} با رسیدن سال 2050 اگر هیچ تمدنی به شرایط دیگر پیروزی نرسیده باشد، امتیازات محاسبه خواهند شد و برنده مشخص خواهد شد. البته پس از رسیدن به این نقطه می‌توانید به بازی ادامه دهید ولی برنده‌‌ی دیگری وجود نخواهد داشت.
\end{enumerate}
\subsubsection*{{\titr امتیازدهی}}
\addcontentsline{toc}{subsubsection}{{\fehrestContent امتیازدهی}}
امتیازها در دو موقعیت مهم هستند: اول اینکه تا سال 2050 تیمی به پیروزی نرسیده باشد که در این صورت تیم برنده بر اساس امتیازها مشخص خواهد شد. دوم اینکه تیمی از یکی از دو روش دیگر پیروز شده باشد. که در این صورت جایگاه این تیم در تالار مشاهیر بر مبنای امتیازش مشخص خواهد شد.
\\\noindent \textbf{حذف:} در صورت حذف از بازی امتیاز شما صفر خواهد بود.
\\\noindent \textbf{پیروزی:} اگر پیش از سال 2050 پیروز شوید امتیاز شما چند برابر خواهد شد. هرچه زودتر پیروز شوید، امتیاز شما در ضریب بزرگتری ضرب خواهد شد.

\subsubsection*{{\titr کسب امتیاز}}
\addcontentsline{toc}{subsubsection}{{\fehrestContent کسب امتیاز}}
از روش‌های زیر می‌توانید در بازی امتیاز کسب کنید:
\begin{itemize}
	\item وسعت سرزمینتان: بر اساس تعداد کاشی‌هایی که در مرزهای شما قرار دارد امتیاز به شما تعلق می‌گیرد. این کم‌اهمیت‌ترین معیار امتیازدهی است.
	\item تعداد شهرهای امپراتوری شما
	\item جمعیت شما
	\item تعداد فناوری‌هایی که به دست آورده اید.
	\item تعداد فناوری‌های آینده که در اختیار دارید.
	\item تعداد عجایبی که ساخته‌اید. این معیار مهمترین معیار در امتیازدهی است.
	\item اندازه‌ی نقشه: اندازه‌ی نقشه‌ای که انتخاب می‌کنید روی امتیازهای وسعت زمین، تعداد شهرها و جمعیت تاثیر دارد
\end{itemize}
\noindent \textbf{امتیاز فعلی شما:}  در طول بازی می‌توانید در پنل دیپلماسی امتیازات تمام بازیکنان را ببینید. واضح است که امتیازات در طول بازی ثابت نیستند.

\subsection*{{\titr Charts}}
\addcontentsline{toc}{subsection}{{\fehrestContent Charts}}
در این داک (توضیحات بازی) همه انواع زمین‌ها، یگان‌ها، تکنولوژی‌ها و ... برای اینکه فضای زیادی گرفته نشود، نیامده است. شما می‌توانید تمامی این موارد را که شما باید پیاده سازی کنید می‌توانید در این 
 \href{https://docs.google.com/document/d/1nc7O3lcWN0OBdwXdwG6vOVuhLKdcxOnMXKr27-Cr8h8/edit?usp=sharing}{\textcolor{blue}{لینک}}
 ببینید.
\end{document}