% !TEX program = xelatex
\documentclass[]{article}
\usepackage{float}
\usepackage{graphicx}
\usepackage[svgnames]{xcolor} 
\usepackage{fancyhdr}
\usepackage{fancyvrb}
\usepackage{forest}
\usepackage{tocloft}
\usepackage[hidelinks]{hyperref}
\usepackage{enumitem}
\usepackage[many]{tcolorbox}
\usepackage{listings }
\usepackage[a4paper, total={6in, 8in} , top = 2cm,bottom = 4cm]{geometry}
%\usepackage[a4paper, total={6in, 8in}]{geometry}
\usepackage{afterpage}
\usepackage{amssymb}
\usepackage{pdflscape}
\usepackage{textcomp}
\usepackage{xecolor}
\usepackage{rotating}
\usepackage[Kashida]{xepersian}
\usepackage[T1]{fontenc}
\usepackage{tikz}
\usepackage[utf8]{inputenc}
\usepackage{PTSerif} 
\usepackage{seqsplit}
\usepackage{changepage}

\usepackage{listings}
\usepackage{xcolor}
\usepackage{sectsty}

\setcounter{secnumdepth}{0}
 
\definecolor{codegreen}{rgb}{0,0.6,0}
\definecolor{codegray}{rgb}{0.5,0.5,0.5}
\definecolor{codepurple}{rgb}{0.58,0,0.82}
\definecolor{backcolour}{rgb}{0.95,0.95,0.92}
\definecolor{blanchedalmond}{rgb}{1.0, 0.92, 0.8}
\definecolor{brilliantlavender}{rgb}{0.96, 0.73, 1.0}
 
\NewDocumentCommand{\codeword}{v}{
\texttt{\textcolor{blue}{#1}}
}
\newcommand{\link}[2]{\href{#1}{\textcolor{blue}{#2}}}
\lstset{language=java,keywordstyle={\bfseries \color{blue}}}

\lstdefinestyle{mystyle}{
    backgroundcolor=\color{backcolour},   
    commentstyle=\color{codegreen},
    keywordstyle=\color{magenta},
    numberstyle=\tiny\color{codegray},
    stringstyle=\color{codepurple},
    basicstyle=\ttfamily\normalsize,
    breakatwhitespace=false,         
    breaklines=true,                 
    captionpos=b,                    
    keepspaces=true,                 
    numbers=left,                    
    numbersep=5pt,                  
    showspaces=false,                
    showstringspaces=false,
    showtabs=false,                  
    tabsize=2
}

\lstset{style=mystyle}

 \settextfont[BoldFont={XB Zar bold.ttf}]{XB Zar.ttf}


\setlatintextfont[Scale=1.0,
 BoldFont={LiberationSerif-Bold.ttf}, 
 ItalicFont={LiberationSerif-Italic.ttf}]{LiberationSerif-Regular.ttf}





\newcommand{\inputsample}[1]{
    ~\\
    \textbf{ورودی نمونه}
    ~\\
    \begin{tcolorbox}[breakable,boxrule=0pt]
        \begin{latin}
            \large{
                #1
            }
        \end{latin}
    \end{tcolorbox}
}

\newcommand{\outputsample}[1]{
    ~\\
    \textbf{خروجی نمونه}

    \begin{tcolorbox}[breakable,boxrule=0pt]
        \begin{latin}
            \large{
                #1
            }
        \end{latin}
    \end{tcolorbox}
}

\newtcolorbox{mybox}[2][]{colback=red!5!white,
colframe=red!75!black,fonttitle=\bfseries,
colbacktitle=red!85!black,enhanced,
attach boxed title to top center={yshift=-2mm},
title=#2,#1}

\newenvironment{changemargin}[2]{%
\begin{list}{}{%
\setlength{\topsep}{0pt}%
\setlength{\leftmargin}{#1}%
\setlength{\rightmargin}{#2}%
\setlength{\listparindent}{\parindent}%
\setlength{\itemindent}{\parindent}%
\setlength{\parsep}{\parskip}%
}%
\item[]}{\end{list}}


\definecolor{foldercolor}{RGB}{124,166,198}
\definecolor{sectionColor}{HTML}{ff5e0e}
\definecolor{subsectionColor}{HTML}{008575}

\definecolor{listColor}{HTML}{00d3b9}

\definecolor{umlrelcolor}{HTML}{3c78d8}

\definecolor{subsubsectionColor}{HTML}{3c78d8}

\defpersianfont\authorFont[Scale=0.9]{XB Zar bold.ttf}

\defpersianfont\titr[Scale=1.5]{Lalezar-Regular.ttf}

\defpersianfont\fehrest[Scale=1.2]{Lalezar-Regular.ttf}

\defpersianfont\fehrestTitle[Scale=3.0]{Lalezar-Regular.ttf}

\defpersianfont\fehrestContent[Scale=1.2]{XB Zar bold.ttf}


\sectionfont{\color{sectionColor}}  % sets colour of sections
\subsectionfont{\color{subsectionColor}}  % sets colour of sections
\subsubsectionfont{\color{subsubsectionColor}}


\renewcommand{\labelitemii}{$\circ$}


\renewcommand{\baselinestretch}{1.1}


\renewcommand{\contentsname}{فهرست}

\renewcommand{\cfttoctitlefont}{\fehrestTitle}


\renewcommand\cftsecfont{\color{sectionColor}\fehrestContent\selectfont}
\renewcommand\cftsubsecfont{\color{subsectionColor}\fehrestContent\selectfont}
\renewcommand\cftsubsubsecfont{\color{subsubsectionColor}\fehrestContent\selectfont}
%\renewcommand{\cftsecpagefont}{\color{sectionColor}}

\setlength{\parskip}{1.2pt}

\begin{document}


%%% title pages
\begin{titlepage}
\begin{center}


\textbf{ \Huge{به نام خدا} }
        
\vspace{0.2cm}

\includegraphics[width=0.4\textwidth]{sharif1.png}\\
\vspace{0.2cm}
\textbf{ \Huge{\emph درس برنامه‌سازی پیشرفته} }\\
\vspace{0.25cm}
\textbf{ \Large{ توضیحات پایگاه داده} }
\vspace{0.2cm}
       
 
      \large \textbf{دانشکده مهندسی کامپیوتر}\\\vspace{0.1cm}
    \large   دانشگاه صنعتی شریف\\\vspace{0.2cm}
       \large   ﻧﯿﻢ سال دوم 01-00 \\\vspace{0.10cm}
      \noindent\rule[1ex]{\linewidth}{1pt}
استاد:\\
    \textbf{{دکتر محمدامین فضلی}}



    \vspace{0.20cm}


    

\end{center}
\end{titlepage}
%%% title pages


%%% header of pages
\newpage
\pagestyle{fancy}
\fancyhf{}
\fancyfoot{}
\cfoot{\thepage}
\lhead{پایگاه داده}
\rhead{\includegraphics[width=0.1\textwidth]{sharif.png}\\
دانشکده مهندسی کامپیوتر
}
\chead{پروژه برنامه‌سازی پیشرفته}
%%% header of pages
\renewcommand{\headrulewidth}{2pt}

\KashidaOff



\tableofcontents

\newpage

 \Large \textbf{\\
}



\section*{{\titr پایگاه‌داده}}
\addcontentsline{toc}{section}{{\fehrestContent پایگاه‌داده}}
\subsection*{{\titr پایگاه‌داده چیست؟}}
\addcontentsline{toc}{subsection}{{\fehrestContent پایگاه‌داده چیست؟}}
بانک اطلاعاتی یا همان پایگاه‌داده (Database) مجموعه‌ای سازمان‌یافته از داده‌ها است که از ذخیره‌سازی الکترونیکی و ایجاد تغییر در داده‌ها پشتیبانی می‌کند. مدیریت داده‌ها به وسیله دیتابیس بسیار آسان می‌شود. برای مدیریت داده‌ها در دیتابیس از سیستم مدیریت پایگاه داده \lr{(Database Management System)}  یا همان DBMS استفاده می‌شود.
\subsection*{{\titr در یک دیتابیس با چه مفاهیمی سروکار داریم؟}}
\addcontentsline{toc}{subsection}{{\fehrestContent در یک دیتابیس با چه مفاهیمی سروکار داریم؟}}
\textbf{داده (Data)}: داده‌ها نمودی از مفاهیم، معلومات، وقایع و پدیده‌ها هستند که از طریق مشاهده یا تحقیق به دست می‌آیند. به طور کلی در حوزه پایگاه داده، داده‌ها در دو نوع رابطه‌ای (Relational) و غیر رابطه‌ای (Non-relational) گروه‌بندی می‌شوند. اغلب اپلیکیشن‌های مدرن حجم وسیعی از هر دو نوع داده‌ها را مورد استفاده قرار می‌دهند.\\
\textbf{موجودیت (Entity)}: موجودیت همان فرد، شی یا پدیده‌ای است که درباره‌اش اطلاعات جمع‌آوری شده است.\\
\textbf{صفت خاصه (Attribute)}: هر ویژگی‌ای که یک موجودیت را از موجودیت دیگر جدا کند، یک صفت خاصه محسوب می‌شود.\\
\textbf{جدول (Table)}: جدول‌ها مهم‌ترین سطح یک دیتابیس محسوب می‌شوند. هر جدول سطر و ستون‌هایی دارد که داده‌ها در آن ذخیره سازی، دسته بندی و ساماندهی می‌شوند.
\subsection*{{\titr انواع پایگاه‌داده}}
\addcontentsline{toc}{subsection}{{\fehrestContent انواع پایگاه‌داده}}
دیتابیس‌ها انواع مختلفی دارند که هر یک بر اساس نیاز به کار می‌روند و از جمله آن‌ها می‌توان به دیتابیس‌های \lr{NoSQL}، بانک‌های اطلاعاتی رابطه‌ای و پایگاه داده شی‌گرا اشاره کرد.\\
دیتابیس رابطه‌ای بر مبنای دو اصطلاح با نام‌های نمونه (Instance) و الگو (Schema) طراحی شده و نمونه آن یک جدول به حساب می‌آید که سطرها یا ستون‌هایی دارد. از سوی دیگر، الگو یا همان اسکیما در دیتابیس رابطه‌ای، تعیین کننده مواردی مانند نام رابطه، نوع هر ستون و سایر موارد مرتبط با ساختار آن است.\\
درحقیقت، دیتابیس‌ها با استفاده از زبان‌های مختلفی ایجاد می‌شوند و در این میان زبان SQL \lr{(Structured Query Language)} از همه معروف‌تر و رایج‌تر است.\\
\textbf{اوراکل \lr{(Oracle)}}: پایگاه داده اوراکل یک سیستم مدیریت پایگاه داده تجاری است. در اوراکل، از فناوری پایگاه داده در مقیاس‌های سازمانی و همراه با ویژگی‌های قدرتمند و خاص استفاده می‌شود. ذخیره‌سازی می‌توان به صورت درون سازمانی یا در فضای ابری انجام شود.\\
\textbf{\lr{MySQL}}: یک سیستم مدیریت پایگاه داده رابطه‌ای است که معمولاً همراه با سیستم‌های مدیریت محتوای (CMS) متن باز و پلتفرم‌های گسترده مختلفی مثل فیسبوک، توییتر و یوتیوب مورد استفاده قرار می‌گیرد.\\
\textbf{\lr{Microsoft SQL Server}}: سیستم \lr{SQL Server} را شرکت مایکروسافت طراحی کرده است. این سیستم یک دیتابیس تجاری است که در سیستم‌های مبتنی بر ویندوز از آن استفاده می‌شود.\\
\textbf{\lr{Microsoft Access / Excel}}: اکسل و اکسس هم از برنامه‌های نام آشنای شرکت مایکروسافت هستند که خیلی‌ها برای ذخیره و پردازش داده‌هایشان از این برنامه‌ها استفاده می‌کنند.\\
از معرف‌ترین پایگاه‌داده‌های NoSQL می‌توان به \lr{MongoDB}، \lr{Redis}، \lr{Cassandra} و \lr{HBase} اشاره کرد. در این نوع دیتابیس‌ها، به جای استفاده از زبان \lr{SQL}، از زبان‌های JSON و XML استفاده می‌شود.\\
دیتابیس‌های SQL به طور کلی با عنوان بانک اطلاعاتی رابطه‌ای (RDBMS) و دیتابیس‌های NoSQL به عنوان بانک اطلاعاتی غیر رابطه‌ای و توزیع شده شناخته می‌شوند.\\
طراحی دیتابیس‌های NoSQL به گونه‌ای است که با استفاده از آن‌ها، امکان گنجاندن طیف وسیعی از مدل‌های داده فراهم می‌شود. دیتابیس \lr{NoSQL}، به عنوان یک روش جایگزین برای دیتابیس‌های رابطه‌ای سنتی طراحی شده است. استفاده از دیتابیس NoSQL زمانی مفید است که مجموعه وسیعی از داده‌های توزیع شده وجود داشته باشد، اما دیتابیس‌های SQL به دلیل استفاده از جدول برای داده‌های ساختار یافته مناسب‌ترند. برای محیط‌هایی با تراکنش بسیار زیاد، دیتابیس‌های SQL توانایی بهتری را از خودشان نشان می‌دهند که این مورد برای اپلیکیشن‌هایی با حجم داده بالا بسیار مناسب است.
\subsection*{{\titr SQLite}}
\addcontentsline{toc}{subsection}{{\fehrestContent SQLite}}
SQLite یک برنامه مدیریت دیتابیس مبتنی بر زبان استاندارد SQL هست.\\
بر خلاف مدل‌های مرسوم دیتابیس که به صورت Client/Server هستند و نیاز به نصب و پیکربندی دارند‌،SQLite  تنها یک برنامه مدیریت دیتابیس مستقل است که نیازی به نصب و پیکربندی ندارد و به کارگیری آن به صورت ضمیمه شده در سیستم‌های مختلف مهم ترین هدف از عرضه آن است.\\
SQLite یک دیتابیس کوچک (با حجم حدود 500 کیلوبایت) است که با زبان C در قالب یک کتابخانه نوشته شده و از پایگاه داده‌های RDBMS به حساب می‌آید.\\
به راحتی با اضافه کردن dependency آخرین نسخه SQLite از \link{https://mvnrepository.com/artifact/org.xerial/sqlite-jdbc}{اینجا} می‌توانید پروژه خود را به دیتابیس متصل کنید.\\
برای مطالعه نمونه پیکربندی و پیاده سازی دیتابیس SQLite در جاوا به \link{https://www.sqlitetutorial.net/sqlite-java/}{اینجا} بروید.\\
\lr{SQLite Expert} ابزاری قدرتمند می‌باشد که به شما این امکان را می‌دهد تا بتوانید به آسانی دیتابیس‌های SQLite خود را مدیریت کرده و دید بهتری نسبت به اینکه چگونه دیتابیس شما عمل می‌کند، بدست آورید. این نرم افزار، مدیریت دیتابیس و نگه‌داری از آن را در محیطی یکتا و یکپارچه و با واسطی گرافیکی تمیز، ترکیب می‌کند. با استفاده از \lr{SQLite Expert} قادر خواهید بود جداول را به صورت بصری و بدون نوشتن حتی یک خط SQL ویرایش کنید. آخرین نسخه این نرم افزار را می‌توانید از \link{https://soft98.ir/software/programming/15259-SQLite-Expert.html}{اینجا} دانلود کنید.
\subsection*{{\titr Redis}}
\addcontentsline{toc}{subsection}{{\fehrestContent Redis}}
Redis براساس تعریف آن در redis.io مخفف عبارت \lr{Remote Dictionary Server} است. در واقع ردیس یک نوع ساختمانِ داده است که در RAM قرار می‌گیرد و اطلاعات به صورت موقتی در آن ذخیره می‌شوند.\\
در ردیس، خبری از ستون‌ها، ردیف‌ها، جدول‌ها و توابع نیست. در عوض، ردیس از ساختمان داده‌هایی مثل \lr{Hash}، \lr{Set}، \lr{List}، \lr{String} و... برای مرتب کردن اطلاعات استفاده می‌کند.\\
ردیس داده‌ها را با سیستم Key-Value نگهداری می‌کند و به لطف این ویژگی از آنجایی که رابطه پیچیده‌ای میان داده‌ها ایجاد نمی‌شود، دسترسی و بازیابی این اطلاعات بسیار ساده‌تر خواهد شد.\\
برای پی بردن به کاربرد ردیس کافی است نگاهی به مارکت‌های فروش اپلیکیشن برای گوشی‌های هوشمند داشته باشید تا متوجه شوید این روزها تا چه اندازه اپ‌های موبایل مخاطب دارند. از طرف دیگر کسب‌و‌کارهای زیادی را شاهد هستیم که تجارت خود را به وب‌سایت‌ها منتقل کرده‌اند و از این طریق به موفقیت‌های بیشتری دست ‌پیدا کرده‌‌اند.\\
این امر باعث شده تا برنامه‌هایی که در سمت سرور \lr{(Server Side)}، اجرا می‌شوند مخاطبین بیشتری پیدا کنند. ساز و کار این برنامه‌ها باید به نوعی باشد که بتوانند پاسخگوی حجم بالای درخواست‌هایی سمت سرور باشند. از طرف دیگر این درخواست‌ها باید با سرعت پاسخ داده شوند تا شاهد ترافیک داده‌ها نباشیم. در این بین بهترین راه‌حل استفاده از دیتابیس‌های NoSQL است که ردیس یکی از پرطرفدارترین و کارآمدترین دیتابیس‌های NoSQL است.\\
Redisson یک \lr{Redis Client} برای جاواست که یک شبکه داده در حافظه تشکیل و سرویس‌های توزیع شده را با پشتیبانی Redis ارائه می‌دهد. مدل داده‌های توزیع شده در حافظه آن امکان اشتراک گذاری object ها و سرویس‌های دامنه را در بین برنامه‌ها و سرورها فراهم می‌کند.\\
برای مطالعه نمونه پیکربندی و پیاده سازی Redis در جاوا می‌توانید به \link{https://www.baeldung.com/redis-redisson}{اینجا} بروید.

\end{document}







