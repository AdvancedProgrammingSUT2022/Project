\documentclass[]{article}
\usepackage{graphicx}
\usepackage[svgnames]{xcolor} 
\usepackage{fancyhdr}
\usepackage{fancyvrb}
\usepackage{forest}
\usepackage{tocloft}
\usepackage[hidelinks]{hyperref}
\usepackage{enumitem}
\usepackage[many]{tcolorbox}
\usepackage{listings }
\usepackage[a4paper, total={6in, 8in} , top = 2cm,bottom = 4cm]{geometry}
%\usepackage[a4paper, total={6in, 8in}]{geometry}
\usepackage{afterpage}
\usepackage{amssymb}
\usepackage{pdflscape}
\usepackage{textcomp}
\usepackage{xecolor}
\usepackage{rotating}
\usepackage[Kashida]{xepersian}
\usepackage[T1]{fontenc}
\usepackage{tikz}
\usepackage[utf8]{inputenc}
\usepackage{PTSerif} 
\usepackage{seqsplit}
\usepackage{changepage}


\usepackage{listings}
\usepackage{xcolor}
\usepackage{sectsty}

\setcounter{secnumdepth}{0}
 
\definecolor{codegreen}{rgb}{0,0.6,0}
\definecolor{codegray}{rgb}{0.5,0.5,0.5}
\definecolor{codepurple}{rgb}{0.58,0,0.82}
\definecolor{backcolour}{rgb}{0.95,0.95,0.92}
\definecolor{blanchedalmond}{rgb}{1.0, 0.92, 0.8}
\definecolor{brilliantlavender}{rgb}{0.96, 0.73, 1.0}
 
\NewDocumentCommand{\codeword}{v}{
\texttt{\textcolor{blue}{#1}}
}
\lstset{language=java,keywordstyle={\bfseries \color{blue}}}

\lstdefinestyle{mystyle}{
    backgroundcolor=\color{backcolour},   
    commentstyle=\color{codegreen},
    keywordstyle=\color{magenta},
    numberstyle=\tiny\color{codegray},
    stringstyle=\color{codepurple},
    basicstyle=\ttfamily\normalsize,
    breakatwhitespace=false,         
    breaklines=true,                 
    captionpos=b,                    
    keepspaces=true,                 
    numbers=left,                    
    numbersep=5pt,                  
    showspaces=false,                
    showstringspaces=false,
    showtabs=false,                  
    tabsize=2
}

\lstset{style=mystyle}

 \settextfont[BoldFont={XB Zar bold.ttf}]{XB Zar.ttf}


\setlatintextfont[Scale=1.0,
 BoldFont={LiberationSerif-Bold.ttf}, 
 ItalicFont={LiberationSerif-Italic.ttf}]{LiberationSerif-Regular.ttf}





\newcommand{\inputsample}[1]{
    ~\\
    \textbf{ورودی نمونه}
    ~\\
    \begin{tcolorbox}[breakable,boxrule=0pt]
        \begin{latin}
            \large{
                #1
            }
        \end{latin}
    \end{tcolorbox}
}

\newcommand{\outputsample}[1]{
    ~\\
    \textbf{خروجی نمونه}

    \begin{tcolorbox}[breakable,boxrule=0pt]
        \begin{latin}
            \large{
                #1
            }
        \end{latin}
    \end{tcolorbox}
}

\newtcolorbox{mybox}[2][]{colback=red!5!white,
colframe=red!75!black,fonttitle=\bfseries,
colbacktitle=red!85!black,enhanced,
attach boxed title to top center={yshift=-2mm},
title=#2,#1}

\newenvironment{changemargin}[2]{%
\begin{list}{}{%
\setlength{\topsep}{0pt}%
\setlength{\leftmargin}{#1}%
\setlength{\rightmargin}{#2}%
\setlength{\listparindent}{\parindent}%
\setlength{\itemindent}{\parindent}%
\setlength{\parsep}{\parskip}%
}%
\item[]}{\end{list}}


\definecolor{foldercolor}{RGB}{124,166,198}
\definecolor{sectionColor}{HTML}{ff5e0e}
\definecolor{subsectionColor}{HTML}{008575}

\definecolor{listColor}{HTML}{00d3b9}

\definecolor{umlrelcolor}{HTML}{3c78d8}

\definecolor{subsubsectionColor}{HTML}{3c78d8}

\defpersianfont\authorFont[Scale=0.9]{XB Zar bold.ttf}

\defpersianfont\titr[Scale=1.5]{Lalezar-Regular.ttf}

\defpersianfont\fehrest[Scale=1.2]{Lalezar-Regular.ttf}

\defpersianfont\fehrestTitle[Scale=3.0]{Lalezar-Regular.ttf}

\defpersianfont\fehrestContent[Scale=1.2]{XB Zar bold.ttf}


\sectionfont{\color{sectionColor}}  % sets colour of sections
\subsectionfont{\color{subsectionColor}}  % sets colour of sections
\subsubsectionfont{\color{subsubsectionColor}}


\renewcommand{\labelitemii}{$\circ$}


\renewcommand{\baselinestretch}{1.1}


\renewcommand{\contentsname}{فهرست}

\renewcommand{\cfttoctitlefont}{\fehrestTitle}


\renewcommand\cftsecfont{\color{sectionColor}\fehrestContent\selectfont}
\renewcommand\cftsubsecfont{\color{subsectionColor}\fehrestContent\selectfont}
\renewcommand\cftsubsubsecfont{\color{subsubsectionColor}\fehrestContent\selectfont}
%\renewcommand{\cftsecpagefont}{\color{sectionColor}}

\setlength{\parskip}{1.2pt}

\begin{document}


%%% title pages
\begin{titlepage}
\begin{center}

\textbf{ \Huge{به نام خدا} }
        
\vspace{0.2cm}

\includegraphics[width=0.4\textwidth]{sharif1.png}\\
\vspace{0.2cm}
\textbf{ \Huge{\emph درس برنامه‌سازی پیشرفته} }\\
\vspace{0.25cm}
\textbf{ \Large{ فاز صفر پروژه} }
\vspace{0.2cm}
       
 
      \large \textbf{دانشکده مهندسی کامپیوتر}\\\vspace{0.1cm}
    \large   دانشگاه صنعتی شریف\\\vspace{0.2cm}
       \large   ﻧﯿﻢ سال دوم 01-00 \\\vspace{0.10cm}
      \noindent\rule[1ex]{\linewidth}{1pt}
استاد:\\
    \textbf{{دکتر محمدامین فضلی}}



    \vspace{0.20cm}

   مهلت ارسال:\\
    \textbf{{26 فروردین - }}
    \textbf{{ساعت 23:59:59}}

    \vspace{0.10cm}



\end{center}
\end{titlepage}
%%% title pages


%%% header of pages
\newpage
\pagestyle{fancy}
\fancyhf{}
\fancyfoot{}
\cfoot{\thepage}
\lhead{فاز صفر}
\rhead{\includegraphics[width=0.1\textwidth]{sharif.png}\\
دانشکده مهندسی کامپیوتر
}
\chead{پروژه برنامه‌سازی پیشرفته}
%%% header of pages
\renewcommand{\headrulewidth}{2pt}

\KashidaOff


 \Large \textbf{\\\\
}






\section*{{\titr مقدمه}}

در این فاز شما باید مقدمات پروژه را حاضر کنید. این مقدمات شامل ابزارهای مورد استفاده در پروژه و همچنین طراحی معماری پروژه توسط دیاگرام \lr{UML} است. مراحل این فاز عبارت اند از:

\begin{enumerate}

\item
\hyperref[subsec:github]{\textcolor{blue}{راه اندازی مخزن \lr{GitHub}}}

\item
\hyperref[subsec:GitHub Issue]{\textcolor{blue}{راه‌اندازی \lr{Github Issue}}}

\item
\hyperref[subsec:uml]{\textcolor{blue}{طراحی \lr{UML} برای منطق پروژه}}


\end{enumerate}

در بخش بعد، هر یک از این موارد شرح داده شده‌اند.

\newpage
\section*{{\titr کارهایی که باید در فاز صفر انجام دهید}}


\subsection*{{\titr راه‌اندازی مخزن GitHub}}
\label{subsec:github}

همانطور که می‌دانید برای پروژه لازم است با گروهتان بر روی یک مخزن \lr{(repository)} گیت فعالیت کنید. برای ساختن این مخزن، کافیست وارد
 \href{https://classroom.github.com/a/536Wx2Tr}{\textcolor{blue}{\underline{این لینک}}} 
 شوید.

ابتدا با لیستی مواجه می‌شوید که شماره دانشجویی تمام افراد در آن موجود است. شماره دانشجویی خود را بیابید و بر روی آن کلیک کنید.

در صفحه‌ی بعد شما باید تیم خود را انتخاب کنید. چنانچه نفر اول گروه خود (سازنده‌ی مخزن) هستید، باید یک تیم بسازید. تنها شماره‌ی گروه پروژه خود را در قسمت نام تیم وارد کنید و تیم را بسازید. لطفا نام تیم خود را براساس شماره تیم با فرمت زیر انتخاب کنید:
\begin{flushleft}
\lr{Group-\#\#}
\end{flushleft}


مثلا اگر تیم $1$ هستید، نام تیم خود را \lr{Group-01} انتخاب کنید. نفرات بعدی گروه شما، باید تیم‌شان را از لیست تیم‌های موجود انتخاب کنند و نیازی به ایجاد تیم ندارند.

پس از این مراحل مخزن شما آماده خواهد شد و لینک آن در اختیارتان قرار خواهد گرفت. توجه کنید که ممکن است بسته به وضعیت سرورهای گیتهاب بین $1$ ثانیه تا حدود $1$ دقیقه آماده شدن مخزن شما طول بکشد. اما به طور معمول در زمان کمتر از $5$ ثانیه مخزن آماده می‌شود. پس از آماده شدن این مخزن، هر یک از اعضای پروژه باید نام و شماره دانشجویی خود را به فایل \lr{README.md} اضافه کند. 

\newpage

\subsection*{{\titr راه‌اندازی Issue Github}}

\label{subsec:GitHub Issue}

در این پروژه، شما باید وظایف هر شخص را در  \lr{Issue} مخزن گیت هاب مشخص کرده و پس از انجام آن کار توسط هر شخص، گزینه تمام شده برای تسک مربوط به آن را تیک بزنید.

هریک از اعضای تیم پروژه باید تسک‌های مربوط به خود را قبل از شروع آن در گیت‌هاب ایشو وارد کند، و سپس آن‌ها را به‌روزرسانی کند.
برای شروع، هر نفر تسک زیر را به تسک‌های خود در ایشو مخزن گیت‌هاب اضافه کند.
\begin{itemize}

\item
کامیت نام و شماره دانشجویی در \lr{README.md} (تسک نفر اول)

\item
کامیت نام و شماره دانشجویی در \lr{README.md} (تسک نفر دوم)

\item
کامیت نام و شماره دانشجویی در \lr{README.md} (تسک نفر سوم)

\end{itemize}

در ادامه‌ی کار نیز باید از این ابزار استفاده کنید ‌و تسک‌های خود را مدیریت کنید. \\

توجه: استفاده از گیت‌هاب ایشو اجباری بوده و در نمره شما تاثیر دارد، همچنین تسک‌های اعضای هر تیم توسط راهنماها و همچنین اعضای تیم طراحی پروژه، در طول بازه انجام پروژه دائما بررسی خواهد شد.\\

برای کار با گیت‌هاب ایشو این
	\href{https://drive.google.com/file/d/1itbPuW17PJt0KDTaxutk6jHH-l1IEZNO/view?usp=sharing}{\textcolor{blue}{فیلم}} 
را ببینید.
\newpage
\subsection*{{\titr طراحی UML برای منطق پروژه}}
\label{subsec:uml}

برای این بخش، شما لازم است مستند فاز اول پروژه را به طور کامل و دقیق مطالعه نمایید. سپس یک \lr{UML} مناسب برای تمام بخش‌های منطق پروژه طراحی کنید.

نکته: برای رسم uml می توانید از هر ابزار دلخواهی استفاده کنید؛ اما دست‌خط و شکل روی کاغذ به هیچ‌وجه قابل قبول نیست. می‌توانید از نرم‌افزارها یا سایت‌های زیر استفاده کنید:

\begin{enumerate}
	
	\item
	
	\href{https://app.diagrams.net/}{\textcolor{blue}{\underline{سایت \lr{Draw.io}}}} 
	
	\item
	
	\href{https://www.lucidchart.com/pages/}{\textcolor{blue}{\underline{سایت Lucidchart}}} 
	
	
	\item
	
	\href{https://products.office.com/en/visio/flowchart-software}{\textcolor{blue}{\underline{نرم افزار \lr{Microsoft Visio}}}} 
	
	
	\item
	
	\href{https://www.umlet.com/changes.htm}{\textcolor{blue}{\underline{نرم افزار UMLet}}} 
	
	
	\item
	
	\href{https://online.visual-paradigm.com/}{\textcolor{blue}{\underline{نرم افزار \lr{Visual Paradigm}}}} 
	
	
	\item
	
	\href{https://www.modelio.org/}{\textcolor{blue}{\underline{نرم افزار Modelio}}} 
	
	

	
	
	
\end{enumerate}


برای آشنایی بیش‌تر با UML به کدنامه مراجعه کنید.



\end{document}

















